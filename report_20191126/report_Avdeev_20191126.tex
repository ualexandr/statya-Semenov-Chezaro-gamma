\documentclass[a4paper,11pt]{article} %размер бумаги устанавливаем А4, шрифт 12пунктов
\usepackage[T2A]{fontenc}
\usepackage[utf8]{inputenc}
\usepackage[english,russian]{babel} %используем русский и английский языки с переносами
\usepackage{amssymb,amsfonts,amsmath,mathtext,enumerate,float,amsthm} %подключаем нужные пакеты расширений
\usepackage[unicode,colorlinks=true,citecolor=black,linkcolor=black]{hyperref}
%\usepackage[pdftex,unicode,colorlinks=true,linkcolor=blue]{hyperref}
\usepackage{indentfirst} % включить отступ у первого абзаца
\usepackage[dvips]{graphicx} %хотим вставлять рисунки?
\graphicspath{{illustr/}}%путь к рисункам

\makeatletter
\renewcommand{\@biblabel}[1]{#1.} % Заменяем библиографию с квадратных скобок на точку:
\makeatother %Смысл этих трёх строчек мне непонятен, но поверим "Запискам дебианщика"

\usepackage{geometry} % Меняем поля страницы.
\geometry{left=2cm}% левое поле
\geometry{right=1cm}% правое поле
\geometry{top=2cm}% верхнее поле
\geometry{bottom=2cm}% нижнее поле

\renewcommand{\theenumi}{\arabic{enumi}.}% Меняем везде перечисления на цифра.цифра
\renewcommand{\labelenumi}{\arabic{enumi}.}% Меняем везде перечисления на цифра.цифра
\renewcommand{\theenumii}{\arabic{enumii}}% Меняем везде перечисления на цифра.цифра
\renewcommand{\labelenumii}{\arabic{enumi}.\arabic{enumii}.}% Меняем везде перечисления на цифра.цифра
\renewcommand{\theenumiii}{\arabic{enumiii}}% Меняем везде перечисления на цифра.цифра
\renewcommand{\labelenumiii}{\arabic{enumi}.\arabic{enumii}.\arabic{enumiii}.}% Меняем везде перечисления на цифра.цифра

\sloppy




\renewcommand\normalsize{\fontsize{14}{25.2pt}\selectfont}

\usepackage[backend=biber,style=gost-numeric,sorting=none]{biblatex}
\addbibresource{../bib/ext.bib}
\addbibresource{../bib/my.bib}
\addbibresource{../bib/Semenov.bib}
\input{../bib/ext.hyphens.bib}


\theoremstyle{plain}
\newtheorem{lemma}{Лемма}[section]
\newtheorem{theorem}[lemma]{Теорема}
\newtheorem{example}[lemma]{Пример}
\newtheorem{property}[lemma]{Свойство}
\newtheorem{remark}[lemma]{Замечание}
\newtheorem{corollary}{Следствие}[lemma]

\usepackage{bbm}


\begin{document}

Через $\ell_\infty$ будем обозначать пространство ограниченных последовательностей с обычной нормой
$
	\|x\| = \sup_{k} |x_k|
$,
через $c$ и $c_0$~--- пространства сходящихся и сходящихся к нулю последовательностей соотв.,
через $\mathbb{N}$~--- множество натуральных чисел,
через $T$~--- оператор сдвига влево, т.е.
$T(x_1,x_2,...) = (x_2,x_3,...)$.


Изучая множество банаховых пределов, Лоренц \cite{lorentz1948contribution}
ввёл пространство почти сходящихся последовательностей $ac$.
Последовательность $x\in\ell_\infty$ называется почти сходящейся,
если существует такое число $t$, что
\begin{equation}
\label{eq:intro_ac}
	\lim_{n\to\infty} \frac{1}{n} \sum_{k=m+1}^{m+n} x_k = t
\end{equation}
равномерно по $m$.
Если $t=0$, то говорят, что последовательность
принадлежит пространству $ac_0$ почти сходящихся к нулю последовательностей.
Свойства пространства $ac$ изучались, в частности, в~\cite{alekhno2006propertiesII,usachev2009_phd_vsu}.

Ниже приводится краткое изложение результатов, полученных в~\cite{our-mz2019ac0}.

\begin{lemma}
Пусть $n_i$~--- строго возрастающая последовательность чисел из $\mathbb{N}$,
\begin{equation*}
	x_k = \left\{\begin{array}{ll}
		1, & \mbox{~если~} k = n_i
		\\
		0  & \mbox{~иначе.~}
	\end{array}\right.
\end{equation*}
Для того, чтобы $x\in ac_0$,
необходимо и достаточно, чтобы
\begin{equation}\label{lim_M(j)/j}
	\lim_{j \to \infty} \frac{1}{j} \liminf_{i\to\infty} (n_{i+j} - n_i) = \infty
	.
\end{equation}
\end{lemma}


Определим нелинейный оператор $\lambda$--срезки $A_\lambda$
для $x = (x_1, x_2, ...)\in\ell_\infty$ как
\begin{equation*}
	(A_\lambda x)_k = \begin{cases}
		1, & \mbox{~если~} x_k \geq \lambda
		\\
		0  & \mbox{~иначе.~}
	\end{cases}
\end{equation*}

\begin{theorem}
\label{thm:lambda_prelim}
Пусть $x\in\ell_\infty$, $x\geq 0$.
Тогда
$
	x\in ac_0
$
если и только если
для любого $\lambda > 0$
выполнено
$
	A_\lambda x \in ac_0
$
.
\end{theorem}


\begin{theorem}
Пусть $x\in\ell_\infty$, $x \geq 0$,
$
	0<\lambda < \limsup_{k\to\infty} x_k
$.
Пусть $\{k: x_k \geq \lambda \} = \{n(\lambda,1),n(\lambda,2),...\}$.
Обозначим
$
	M_{\lambda}(j) = \liminf_{i\to\infty} n(\lambda,i+j) - n(\lambda,i)
$.
Для того, чтобы $x\in ac_0$, необходимо и достаточно, чтобы
для любого $\lambda>0$ было выполнено
$
	\lim_{j \to \infty} \frac{M_{\lambda}(j)}{j} = \infty
$.
\end{theorem}

Требование произвольности $\lambda$ существенно.



Пусть $\rho(x,c)$ и $\rho(x,c_0)$~--- расстояния от $x$ до пространств $c$
и $c_0$ соотв.
Легко видеть, что функция
\begin{equation*}
	\alpha(x) = \limsup_{i\to\infty} \max_{i \leq j \leq 2i} |x_i-x_j|
\end{equation*}
удовлетворяет условию Липшица с константой 2
и для $x\in c$ выполнено
$\alpha(x)=0$.

\begin{lemma}
\label{thm:alpha_x_leq_2_rho_x_c}
	Для любого $x\in\ell_\infty$
	выполнено неравенство
	$
		\alpha(x) \leq 2\rho(x, c)
	$.
\end{lemma}


\begin{lemma}
\label{thm:rho_x_c_leq_alpha_t_s_x}
	Для любых $x\in ac$ и $s\in\mathbb{N}$ верно
	$
		\rho(x,c)\leq \alpha(T^s x)
	$.
%	где $T$~--- оператор сдвига влево, т.е.
%	$T(x_1,x_2,...) = (x_2,x_3,...)$.
\end{lemma}


Последовательность $\alpha(T^s x)$ монотонно невозрастает по $s$,
оставаясь при этом неотрицательной, следовательно, имеет предел.
Из этого наблюдения, а также лемм \ref{thm:alpha_x_leq_2_rho_x_c} и \ref{thm:rho_x_c_leq_alpha_t_s_x}
незамедлительно следует
\begin{theorem}
%\label{thm:rho_x_c_leq_alpha_t_s_x}
	Если $x\in ac$, то
	$
		\frac{1}{2} \alpha(x) \leq \rho(x,c)\leq \lim_{s\to\infty} \alpha(T^s x) \leq \alpha(x)
	$.
\end{theorem}

\begin{corollary}
	Если $x\in ac_0$, то
	$
		\frac{1}{2} \alpha(x) \leq \rho(x,c_0)\leq \lim_{s\to\infty} \alpha(T^s x) \leq \alpha(x)
	$.
\end{corollary}
Все оценки точны, т.е. на некоторых элементах достигается равенство.
\begin{corollary}
	Пусть $x\in ac$.
	Тогда $x\in c$ если и только если $\alpha(x) = 0$.
\end{corollary}

\begin{remark}
	Условие $x\in ac$ в лемме \ref{thm:rho_x_c_leq_alpha_t_s_x} существенно.
	Так, в \cite{our-vzms-2018} для любого $p\in\mathbb{N}$ строится элемент $x=Sy\notin ac$,
	для которого $\alpha(x) = 1/p$ и $\rho(x,c)=1/2$.
\end{remark}


\printbibliography


\end{document}

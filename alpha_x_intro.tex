На пространстве $\ell_\infty$ определяется $\alpha$--функция следующим равенством:
\begin{equation}
	\alpha(x) = \varlimsup_{i\to\infty} \max_{i<j\leqslant 2i} |x_i - x_j|
	.
\end{equation}
Иногда удобнее использовать одно из нижеследующих равносильных определений:
\begin{equation}
	\alpha(x) = \varlimsup_{i\to\infty} \max_{i \leqslant j\leqslant 2i} |x_i - x_j|
	,
\end{equation}
\begin{equation}
	\alpha(x) = \varlimsup_{i\to\infty} \sup_{i<j\leqslant 2i} |x_i - x_j|
	,
\end{equation}
\begin{equation}
	\alpha(x) = \varlimsup_{i\to\infty} \sup_{i \leqslant j\leqslant 2i} |x_i - x_j|
	.
\end{equation}

Легко видеть, что $\alpha$--функция неотрицательна.
Более того, $\alpha$--функция удовлетворяет неравенству треугольника:
\begin{equation}
	\alpha(x+y) \leq \alpha(x) + \alpha(y)
	.
\end{equation}
TODO: доказывать нужно или очевидно?

На пространстве $\ell_\infty$ $\alpha$--функция удовлетворяет условию Липшица:
\begin{equation}
	|\alpha(x) - \alpha(y)| \leq 2 \|x-y\|
	.
\end{equation}
(и эта оценка точна).
TODO: доказывать нужно или очевидно?

TODO: структурировать на утверждения-свойства или не нужно, достаточно номеров формул?

Если $y\in c$, то $\alpha(y) = 0$ и $\alpha(x+y) = \alpha(x)$ для любого $x \in \ell_\infty$.


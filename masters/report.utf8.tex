\documentclass[10pt,pdf,hyperref={unicode}]{beamer}
\usepackage{amsmath}
\usepackage[utf8]{inputenc}
\usepackage[english,russian]{babel}
\usepackage{amsfonts}
\usepackage{amsfonts,amssymb}
\usepackage{amssymb}
\usepackage{latexsym}
\usepackage{euscript}
\usepackage{enumerate}
\usepackage{graphics}
\usepackage{graphicx}
\usepackage{geometry}
\usepackage{wrapfig}

\usepackage{bbm}

\righthyphenmin=2



\usepackage[backend=biber,style=gost-numeric,sorting=none]{biblatex}
\addbibresource{../bib/Semenov.bib}
\addbibresource{../bib/my.bib}
\addbibresource{../bib/ext.bib}

\input{../bib/ext.hyphens.bib}


\usepackage{amsthm}
\theoremstyle{definition}
\newtheorem{llemma}{Лемма}
\newtheorem{ttheorem}[llemma]{Теорема}
\newtheorem{eexample}[llemma]{Пример}
\newtheorem{property}[llemma]{Свойство}
\newtheorem{remark}[llemma]{Замечание}
\newtheorem{ccorollary}{Следствие}[llemma]


%\usetheme{Antibes}
\usefonttheme{professionalfonts} % using non standard fonts for beamer
\usefonttheme[onlymath]{serif} % default family is serif
%\usepackage{fontspec}
%\setmainfont{Liberation Serif}
\begin{document}


\title{Асимптотические характеристики \\ ограниченных последовательностей}
\author{Выполнил: Авдеев Н.Н. \\ Научный руководитель: д.ф.-м.н., проф. Каменский М.И.}
\institute{ Воронежский Государственный Университет \\ Кафедра функционального анализа и операторных уравнений}
\date{Воронеж 2019}

\maketitle

\setbeamertemplate{footline}[frame number]
\setbeamertemplate{navigation symbols}{}

\begin{frame}
	\frametitle{Пространство $\ell_\infty$}
	$\ell_\infty$~--- пространство всех ограниченных последовательностей
	$x=(x_1, x_2, ..., x_n, ...)$
	с нормой
	$$
		\|x\|_{\ell_\infty} = \sup_{k\in\mathbb{N}} |x_k|
	$$
	{Свойства:}

	\begin{itemize}
		\item
			$\ell_\infty$~--- линейное пространство над полем $\mathbb{R}$
		\item
			$\ell_\infty$  несепарабельно
		\item
			$\ell_1 \subset \ell_2 \subset \dots \subset \ell_\infty$
		\item
			$\ell_1^* = \ell_\infty$
		\item
			$\ell_\infty^* \neq \ell_1$
	\end{itemize}
\end{frame}

\begin{frame}\frametitle{Банаховы пределы}
	Банаховым пределом называется функционал $B\in \ell_\infty^*$ такой, что:
	\begin{enumerate}
		\item
			$B \geqslant 0$
		\item
			$B\mathbbm{1} = 1$
		\item
			$B=BT$
	\end{enumerate}
	Здесь $\mathbbm{1}=(1,1,1,1,1,...)$,
	$T$~--- оператор сдвига: $T(x_1, x_2, x_3, ...) = (x_2, x_3, ...)$.

	Простейшие свойства:
	\begin{itemize}
		\item
			$\|B\|_{\ell_\infty^*} = 1$
		\item
			$Bx = \lim_{n\to\infty} x_n$ для любого $x=(x_1, x_2, ...) \in c$,
			\\
			где $c$~--- множество сходящихся последовательностей.

			Таким образом,
			банахов предел~--- естественное обобщение понятия предела сходящейся последовательности
			на все ограниченные последовательности.
	\end{itemize}
\end{frame}

\begin{frame}\frametitle{Теорема Лоренца}
	Для заданного $r\in\mathbb{R}$ равенство $Bx=r$ выполнено для всех $B\in\mathfrak{B}$
	тогда и только тогда, когда
	\begin{equation*}
		\lim_{n\to\infty} \frac{1}{n} \sum_{k=m+1}^{m+n} x_k = r
	\end{equation*}
	сходится равномерно по всем $m\in\mathbb{N}$.

	Множество всех таких $x \in \ell_\infty$ обозначается $ac$.
\end{frame}



\begin{frame}
	\frametitle{Критерий принадлежности $ac_0$~\cite{our-mz2019ac0}}


	\begin{llemma}
		Пусть $n_i$~--- строго возрастающая последовательность чисел из $\mathbb{N}$,
		\begin{equation*}
			x_k = \left\{\begin{array}{ll}
				1, & \mbox{~если~} k = n_i
				\\
				0  & \mbox{~иначе.~}
			\end{array}\right.
		\end{equation*}
		Для того, чтобы $x\in ac_0$,
		необходимо и достаточно, чтобы
		\begin{equation}\label{lim_M(j)/j}
			\lim_{j \to \infty} \frac{M(j)}{j} = \infty
			,
		\end{equation}
		где
		\begin{equation*}
			%\quad\mbox{~где~\quad}
			M(j) = \liminf_{i\to\infty} (n_{i+j} - n_i)
			.
		\end{equation*}
	\end{llemma}


\end{frame}



\begin{frame}
	\frametitle{Критерий принадлежности $ac_0$~\cite{our-mz2019ac0}}


	Определим нелинейный оператор $\lambda$--срезки $A_\lambda$
	для $x = (x_1, x_2, ...)\in\ell_\infty$ как
	\begin{equation*}
		(A_\lambda x)_k = \begin{cases}
			1, & \mbox{~если~} x_k \geq \lambda
			\\
			0  & \mbox{~иначе.~}
		\end{cases}
	\end{equation*}

	\begin{ttheorem}
		\label{thm:lambda_prelim}
		Пусть $x\in\ell_\infty$, $x\geq 0$.
		Тогда
		$
			x\in ac_0
		$
		если и только если
		для любого $\lambda > 0$
		выполнено
		$
			A_\lambda x \in ac_0
		$
		.
	\end{ttheorem}

\end{frame}



\begin{frame}
	\frametitle{Критерий принадлежности $ac_0$~\cite{our-mz2019ac0}}



	\begin{ttheorem}
		Пусть $x\in\ell_\infty$, $x \geq 0$,
		\begin{equation*}
			0<\lambda < \limsup_{k\to\infty} x_k
			.
		\end{equation*}
		Пусть $\{k: x_k \geq \lambda \} = \{n(\lambda,1),n(\lambda,2),...\}$.
		Обозначим
		\begin{equation*}
			M_{\lambda}(j) = \liminf_{i\to\infty} n(\lambda,i+j) - n(\lambda,i)
			.
		\end{equation*}
		Для того, чтобы $x\in ac_0$, необходимо и достаточно, чтобы
		для любого $\lambda>0$
		\begin{equation*}
			\lim_{j \to \infty} \frac{M_{\lambda}(j)}{j} = \infty
			.
		\end{equation*}
	\end{ttheorem}


	\begin{remark}
		Требование произвольности $\lambda$ существенно.
		Например, для $x_k = 1$ и $\lambda = 2$ имеем $A_\lambda x = 0 \in ac_0$,
		но $x\notin ac_0$.
	\end{remark}


\end{frame}


\begin{frame}
	\frametitle{Функция $\alpha(x)$ и пространство $ac$~\cite{our-mz2019ac0}}
	Пусть $\rho(x,c)$ и $\rho(x,c_0)$~--- расстояния от $x$ до пространств $c$
	и $c_0$ соотв.
	Легко видеть, что функция
	\begin{equation*}
		\alpha(x) = \limsup_{i\to\infty} \max_{i \leq j \leq 2i} |x_i-x_j|
	\end{equation*}
	удовлетворяет условию Липшица с константой 2
	и для $x\in c$ выполнено
	$\alpha(x)=0$.

	\begin{llemma}
	\label{thm:alpha_x_leq_2_rho_x_c}
		Для любого $x\in\ell_\infty$
		выполнено неравенство
		$
			\alpha(x) \leq 2\rho(x, c)
		$.
	\end{llemma}


	\begin{llemma}
	\label{thm:rho_x_c_leq_alpha_t_s_x}
		Для любого $x\in ac$ и для любого натурального $s$ верно
		$
			\rho(x,c)\leq \alpha(T^s x)
		$,
	%	где $T$~--- оператор сдвига влево, т.е.
	%	$T(x_1,x_2,...) = (x_2,x_3,...)$.
	\end{llemma}

	Обе оценки точны.
	Условие $x\in ac$ существенно.

\end{frame}


\begin{frame}
	\frametitle{Функция $\alpha(x)$ и пространство $ac$~\cite{our-mz2019ac0}}
	\begin{ttheorem}
	%\label{thm:rho_x_c_leq_alpha_t_s_x}
		Для любого $x\in ac$
		\begin{equation*}
			\frac{1}{2} \alpha(x) \leq \rho(x,c)\leq \lim_{s\to\infty} \alpha(T^s x) \leq \alpha(x)
			.
		\end{equation*}
	\end{ttheorem}

	\begin{ccorollary}
		Для любого $x\in ac_0$
		\begin{equation*}
			\frac{1}{2} \alpha(x) \leq \rho(x,c_0)\leq \lim_{s\to\infty} \alpha(T^s x) \leq \alpha(x)
			.
		\end{equation*}
	\end{ccorollary}

	Обе оценки точны~\cite{our-ped-2018-alpha-Tx}.

	\begin{ccorollary}
		Пусть $x\in ac$.
		Тогда $x\in c$ если и только если $\alpha(x) = 0$.
	\end{ccorollary}

\end{frame}


\begin{frame}
	\frametitle{Инвариантные банаховы пределы~\cite{our-ped-2018-inf-dim-ker}}
	\begin{llemma}
		Пусть $Q:\ell_\infty \to \ell_\infty$.
		Для того, чтобы для любого банахова предела $B$
		было выполнено $BQ = B$,
		необходимо и достаточно, чтобы
		\begin{equation}\label{Avdeev_1_ac0}
			I-Q : \ell_\infty \to ac_0,
		\end{equation}
		где $I$~--- тождественный оператор на $\ell_\infty$.
	\end{llemma}

	\begin{eexample}
		\begin{equation*}
			(Qx)_k =
			\begin{cases}
				0,~\mbox{если}~ k = 2^n, n \in\mathbb{N},
				\\
				x_k~\mbox{иначе.}
			\end{cases}
		\end{equation*}

		Выполнено $\dim \ker Q = \infty$ и $Q$ удовлетворяет условию (\ref{Avdeev_1_ac0}).
	\end{eexample}


\end{frame}


\begin{frame}
	\frametitle{Опубликованные работы}
	\printbibliography{}
\end{frame}


\begin{frame}
	\huge\centering
	Спасибо за внимание
\end{frame}

%%%%  ОФОРМЛЕНИЕ СПИСКА ЛИТЕРАТУРЫ %%%
%\smallskip \centerline{\bf Литература}\nopagebreak

%1. {\it Semenov E. M., Sukochev F. A.}
% Invariant Banach limits and applications //Journal of Functional Analysis. – 2010. – Т. 259. – №. 6. – С. 1517-1541.

\end{document}

\documentclass[a4paper,openbib]{report}
\usepackage{amsmath}
\usepackage[utf8]{inputenc}
\usepackage[english,russian]{babel}
\usepackage{amsfonts,amssymb}
\usepackage{latexsym}
\usepackage{euscript}
\usepackage{enumerate}
\usepackage{graphics}
\usepackage[dvips]{graphicx}
\usepackage{geometry}
\usepackage{wrapfig}
\usepackage[colorlinks=true,allcolors=black]{hyperref}



\righthyphenmin=2

\usepackage[14pt]{extsizes}

\geometry{left=3cm}% левое поле
\geometry{right=1cm}% правое поле
\geometry{top=2cm}% верхнее поле
\geometry{bottom=2cm}% нижнее поле

\renewcommand{\baselinestretch}{1.3}

\newcommand{\longcomment}[1]{}

\begin{document}
\clubpenalty=10000
\widowpenalty=10000
\setcounter{page}{2}
\tableofcontents

\chapter{$\alpha$--функция как асимптотическая характеристика ограниченной последовательности}

	\section{$\alpha$--функция и оператор сдвига}
	\documentclass[a4paper,12pt,openbib]{report}
\usepackage{amsmath}
\usepackage[utf8]{inputenc}
\usepackage[english,russian]{babel}
\usepackage{amsfonts}
\usepackage{amsfonts,amssymb}
\usepackage{amssymb}
\usepackage{latexsym}
\usepackage{euscript}
\usepackage{enumerate}
\usepackage{graphics}
\usepackage[dvips]{graphicx}
\usepackage{geometry}
\usepackage{wrapfig}

\geometry{verbose,a4paper,tmargin=1.75cm,bmargin=2.1cm,lmargin=1.75cm,rmargin=1.75cm}

\righthyphenmin=2


\begin{document}

\clubpenalty=10000
\widowpenalty=10000

Напомним, что на пространстве ограниченных последовательностей $\ell_\infty$
определяется $\alpha$--функция равенством
\begin{equation}
	\alpha(x) = \varlimsup_{i\to\infty}\max_{i < j \leqslant 2i} |x_i - x_j|.
\end{equation}

Однако, как выясняется, $\alpha$--функция не инвариантна относительно оператора сдвига
\begin{equation}
	T(x_1,x_2,x_3,...) = (x_2, x_3, ...).
\end{equation}

\paragraph{Пример.}
Пусть
\begin{equation}
	x_k = \begin{cases}
		(-1)^n, & \mbox{~если~} k = 2^n
		\\
		0 & \mbox{~иначе~}
	\end{cases}
\end{equation}

Вычислим $\alpha(x)$.
Заметим сначала, что из принадлежности $x_k\in\{-1,0,1\}$
немедленно следует, что $\alpha(x) \leq 2$.
Оценим теперь $\alpha(x)$ снизу:
\begin{multline}
	\alpha(x)
	=
	\varlimsup_{i\to\infty}\max_{i < j \leqslant 2i} |x_i - x_j|
	\geq
	\\\geq
	\mbox{(переход к частичному верхнему пределу по индексам специального вида $i=2^n$)}
	\geq
	\\\geq
	\varlimsup_{n\to\infty}\max_{2^n < j \leqslant 2^{n+1}} |x_{2^n} - x_j|
	=
	\varlimsup_{n\to\infty}\max_{2^n < j \leqslant 2^{n+1}} |(-1)^n - x_j|
	\geq
	\varlimsup_{n\to\infty} |(-1)^n - x_{2^{n+1}}|
	=
	\\=
	\varlimsup_{n\to\infty} |(-1)^n - (-1)^{n+1}|
	=
	2
\end{multline}

Итак, $\alpha(x) = 2$.
Вычислим теперь $\alpha(Tx)$:
\begin{multline}
	\alpha(Tx)
	=
	\varlimsup_{i\to\infty}~\max_{i < j \leqslant 2i} |(Tx)_i - (Tx)_j|
	=
	\varlimsup_{i\to\infty}~\max_{i < j \leqslant 2i} |x_{i+1} - x_{j+1}|
	=
	\\=
	(\mbox{замена}~k:=i+1, m:=j+1)
	=
	\varlimsup_{k\to\infty}~~\max_{k-1 < m-1 \leqslant 2k-2} |x_k - x_m|
	=
	\varlimsup_{k\to\infty}~~\max_{k < m \leqslant 2k-1} |x_k - x_m|
	=
	\\=
	\max\left\{
		\varlimsup_{k\to\infty, k  =   2^n}~~\max_{k < m \leqslant 2k-1} |x_k - x_m|
		,~~
		\varlimsup_{k\to\infty, k \neq 2^n}~~\max_{k < m \leqslant 2k-1} |x_k - x_m|
	\right\}
	=
	\\=
	\max\left\{
		\varlimsup_{n\to\infty}~~\max_{2^n < m \leqslant 2^{n+1}-1} |x_{2^n} - x_m|
		,~~
		\varlimsup_{k\to\infty, k \neq 2^n}~~\max_{k < m \leqslant 2k-1} |x_k - x_m|
	\right\}
	=
	\\=
	\max\left\{
		\varlimsup_{n\to\infty}~~\max_{2^n < m \leqslant 2^{n+1}-1} |(-1)^n - x_m|
		,~~
		\varlimsup_{k\to\infty, k \neq 2^n}~~\max_{k < m \leqslant 2k-1} |x_k - x_m|
	\right\}
	=
	\\=
	\max\left\{
		\varlimsup_{n\to\infty}~~\max_{2^n < m \leqslant 2^{n+1}-1} |(-1)^n - 0|
		,~~
		\varlimsup_{k\to\infty, k \neq 2^n}~~\max_{k < m \leqslant 2k-1} |x_k - x_m|
	\right\}
	=
	\\=
	\max\left\{
		1
		,~
		\varlimsup_{k\to\infty, k \neq 2^n}~~\max_{k < m \leqslant 2k-1} |x_k - x_m|
	\right\}
	=
	\\=
	\mbox{(если $k \neq 2^n$, то $x_k = 0$)}
	=
	\\=
	\max\left\{
		1
		,~
		\varlimsup_{k\to\infty, k \neq 2^n}~~\max_{k < m \leqslant 2k-1} |0 - x_m|
	\right\}
	=
	1
	.
\end{multline}
Таким образом, $\alpha(Tx) = 1 \neq 2 = \alpha(x)$,
что и требовалось показать.

Верна следующая
\paragraph{Теорема.}
Для любого $x \in \ell_\infty$ выполнено неравенство $\alpha(Tx)\leq \alpha(x)$.

\paragraph{Доказательство.}
\begin{multline}
	\alpha(Tx)
	=
	\varlimsup_{i\to\infty}~\max_{i < j \leqslant 2i} |(Tx)_i - (Tx)_j|
	=
	\varlimsup_{i\to\infty}~\max_{i < j \leqslant 2i} |x_{i+1} - x_{j+1}|
	=
	\\=
	(\mbox{замена}~k:=i+1, m:=j+1)
	=
	\\=
	\varlimsup_{k\to\infty}~~\max_{k-1 < m-1 \leqslant 2k-2} |x_k - x_m|
	=
	\varlimsup_{k\to\infty}~~\max_{k < m \leqslant 2k-1} |x_k - x_m|
	\leq
	\\ \leq
	\mbox{(переход к максимуму по большему множеству)}
	\leq
	\\ \leq
	\varlimsup_{k\to\infty}~~\max_{k < m \leqslant 2k} |x_k - x_m|
	=
	\alpha(x)
	.
\end{multline}

Более интересна, однако, следующая оценка.
\paragraph{Теорема.}
Для любого $n\in\mathbb{N}$
\begin{equation}
	\alpha(T^n x) \geq \frac{1}{2} \alpha(x)
	.
\end{equation}

\paragraph{Доказательство.}
\begin{multline}
	\alpha(x)
	=
	\varlimsup_{i\to\infty}~~\max_{i < j \leqslant 2i} |x_i - x_j|
	\leq
	\varlimsup_{i\to\infty} \left(\max_{i < j \leqslant 2i-n} |x_i - x_j| + \max_{2i-n < j \leqslant 2i} |x_i - x_j|\right)
	\leq
	\\ \leq
	\varlimsup_{i\to\infty} ~~\max_{i < j \leqslant 2i-n} |x_i - x_j| + \varlimsup_{i\to\infty} ~~\max_{2i-n < j \leqslant 2i} |x_i - x_j|
	=
	\\=
	(\mbox{замена}~k:=i-n, m:=j-n ~\mbox{в первом пределе})
	=
	\\=
	\varlimsup_{k\to\infty} ~~\max_{k+n < m+n \leqslant 2k+2n-n} |x_{k+n} - x_{m+n}| + \varlimsup_{i\to\infty} ~~\max_{2i-n < j \leqslant 2i} |x_i - x_j|
	=
	\\=
	\varlimsup_{k\to\infty} ~~\max_{k < m \leqslant 2k} |x_{k+n} - x_{m+n}| + \varlimsup_{i\to\infty} ~~\max_{2i-n < j \leqslant 2i} |x_i - x_j|
	=
	\\=
	\varlimsup_{k\to\infty} ~~\max_{k < m \leqslant 2k} |(T^n x)_k - (T^n x)_m| + \varlimsup_{i\to\infty} ~~\max_{2i-n < j \leqslant 2i} |x_i - x_j|
	=
	\\=
	\alpha(T^n x) + \varlimsup_{i\to\infty} ~~\max_{2i-n < j \leqslant 2i} |x_i - x_j|
	=
	\\=
	(\mbox{замена}~k:=i-n, m:=j-n ~\mbox{в оставшемся пределе})
	=
	\\=
	\alpha(T^n x) + \varlimsup_{k\to\infty} ~~\max_{2k+n < m+n \leqslant 2k+2n} |x_{k+n} - x_{m+n}|
	=
	\alpha(T^n x) + \varlimsup_{k\to\infty} ~~\max_{2k < m \leqslant 2k+n} |x_{k+n} - x_{m+n}|
	=
	\\=
	\alpha(T^n x) + \varlimsup_{k\to\infty} ~~\max_{2k < m \leqslant 2k+n} |(T^n x)_k - (T^n x)_m|
\end{multline}

XXXXXXXXXXXXXXXX

\begin{multline}
	\varlimsup_{i\to\infty}\max_{i < j \leqslant 2i} |x_i - x_{j-n} + x_{j-n} - x_j|
	\leq
	\varlimsup_{i\to\infty}\max_{i < j \leqslant 2i} (|x_i - x_{j-n}| + |x_{j-n} - x_j|)
	\leq
	\\ \leq
	\varlimsup_{i\to\infty} ( \max_{i < j \leqslant 2i} |x_i - x_{j-n}| + \max_{i < j \leqslant 2i}  |x_{j-n} - x_j|)
	\leq
	\varlimsup_{i\to\infty} \max_{i < j \leqslant 2i} |x_i - x_{j-n}| + \varlimsup_{i\to\infty} \max_{i < j \leqslant 2i}  |x_{j-n} - x_j|
	=
	\\=
	(\mbox{замена}~k:=i-n, m:=j-n ~\mbox{в первом пределе})
	=
	\\=
	\varlimsup_{k\to\infty} \max_{k+n < m+n \leqslant 2k+2n} |x_{k+n} - x_{m}| + \varlimsup_{i\to\infty} \max_{i < j \leqslant 2i}  |x_{j-n} - x_j|
	=
	\\=
	\varlimsup_{k\to\infty} \max_{k < m \leqslant 2k+n} |x_{k+n} - x_{m}| + \varlimsup_{i\to\infty} \max_{i < j \leqslant 2i}  |x_{j-n} - x_j|
\end{multline}


\end{document}


	\section{$\alpha$--функция и семейство операторов $\sigma_n$}
	\documentclass[a4paper,14pt]{article} %размер бумаги устанавливаем А4, шрифт 12пунктов
\usepackage[T2A]{fontenc}
\usepackage[utf8]{inputenc}
\usepackage[english,russian]{babel} %используем русский и английский языки с переносами
\usepackage{amssymb,amsfonts,amsmath,mathtext,cite,enumerate,float,amsthm} %подключаем нужные пакеты расширений
\usepackage[unicode,colorlinks=true,citecolor=black,linkcolor=black]{hyperref}
%\usepackage[pdftex,unicode,colorlinks=true,linkcolor=blue]{hyperref}
\usepackage{indentfirst} % включить отступ у первого абзаца
\usepackage[dvips]{graphicx} %хотим вставлять рисунки?
\graphicspath{{illustr/}}%путь к рисункам

\makeatletter
\renewcommand{\@biblabel}[1]{#1.} % Заменяем библиографию с квадратных скобок на точку:
\makeatother %Смысл этих трёх строчек мне непонятен, но поверим "Запискам дебианщика"

\usepackage{geometry} % Меняем поля страницы.
\geometry{left=2cm}% левое поле
\geometry{right=1cm}% правое поле
\geometry{top=2cm}% верхнее поле
\geometry{bottom=2cm}% нижнее поле

\renewcommand{\theenumi}{\arabic{enumi}}% Меняем везде перечисления на цифра.цифра
\renewcommand{\labelenumi}{\arabic{enumi}}% Меняем везде перечисления на цифра.цифра
\renewcommand{\theenumii}{.\arabic{enumii}}% Меняем везде перечисления на цифра.цифра
\renewcommand{\labelenumii}{\arabic{enumi}.\arabic{enumii}.}% Меняем везде перечисления на цифра.цифра
\renewcommand{\theenumiii}{.\arabic{enumiii}}% Меняем везде перечисления на цифра.цифра
\renewcommand{\labelenumiii}{\arabic{enumi}.\arabic{enumii}.\arabic{enumiii}.}% Меняем везде перечисления на цифра.цифра

\sloppy


\renewcommand\normalsize{\fontsize{14}{25.2pt}\selectfont}

\begin{document}
% !!!
% Здесь начинается реальный ТеХ-код
% Всё, что выше - беллетристика

\paragraph{Теорема.}
$$
	\forall(x\in l_\infty) \forall(n\in\mathbb{N})
	[
		\alpha(\sigma_n x) = \alpha(x)
	]
$$

\paragraph{Доказательство.}
По определению
\begin{equation}
	\alpha(x) = \varlimsup_{i\to\infty} \max_{i<j\leqslant 2i} |x_i - x_j|
\end{equation}

Положим
\begin{equation}
	\alpha_i(x) =
	\max_{i<j\leqslant 2i} |x_i - x_j| =
	\max_{i\leqslant j\leqslant 2i} |x_i - x_j|
\end{equation}

Тогда
\begin{equation}
	\alpha(x) = \varlimsup_{i\to\infty} \alpha_i(x)
\end{equation}

Пусть $y = \sigma_n x$.
Тогда для $k=1, ..., n-1$, $a\in\mathbb{N}$ имеем
\begin{multline}
	\alpha_{an-k}(y) =
	\max_{an-k \leqslant j \leqslant 2an-2k} |y_{an-k} - y_j| =
	\\=
	(\mbox{т.к.}~y_{an-(n-1)}=y_{an-(n-2)}=...=y_{an-k}=...=y_{an-1}=y_{an})=
	\\=
	\max_{an \leqslant j \leqslant 2an-2k} |y_{an} - y_j| \leqslant
	\\ \leqslant
	(\mbox{переходим к максимуму по большему множеству}) \leqslant
	\\ \leqslant
	\max_{an \leqslant j \leqslant 2an} |y_{an} - y_j| =
	\alpha_{an}(y)
\end{multline}

С другой стороны,
\begin{multline}
	\alpha_{an}(y) =
	\max_{an \leqslant j \leqslant 2an} |y_{an} - y_j| =
	\\ =
	(\mbox{т.к.}~y=\sigma_n x,~\mbox{можем рассматривать только}~j=kn)=
	\\ =
	\max_{an \leqslant kn \leqslant 2an} |y_{an} - y_{kn}| =
	\max_{a \leqslant k \leqslant 2a} |y_{an} - y_{kn}| =
	\max_{a \leqslant k \leqslant 2a} |x_a - x_k| =
	\alpha_a(x)
\end{multline}

Таким образом, для $k=1, ..., n-1$, $a\in\mathbb{N}$ имеем соотношения:
\begin{gather}
	\alpha_{an}(y) = \alpha_a(x),
\\
	\alpha_{an-k}(y) \leqslant \alpha_a(x),
\end{gather}
откуда немедленно следует, что
\begin{equation}
	\varlimsup_{i\to\infty} \alpha_i(y) =
	\varlimsup_{i\to\infty} \alpha_i(x),
\end{equation}
т.е.
\begin{equation}
	\alpha(\sigma_n x) = \alpha(x),
\end{equation}
что и требовалось доказать.

\paragraph{Следствие.}
$$
	\alpha(C\sigma_2 x) =
	\alpha(\sigma_2 Cx) =
	\alpha(Cx)
$$

\end{document}


	\section{$\alpha$--функция оператор Чезаро $C$}
	Ниже приводится расширенная версия материала, опубликованного в
\cite{our-vzms-2018}.

На пространстве ограниченных последовательностей $\ell_\infty$ определяется оператор Чезаро $C$
равенством
\begin{equation}
	(Cx)_n = {1}/{n} \cdot \sum_{k=1}^n x_k
	.
\end{equation}

Можно доказать, что верна
\begin{theorem}
	\label{thm:alpha_Cx_leq_alpha_x}
	%TODO: ссылка?
	$\alpha(Cx) \leqslant \alpha(x)$.
\end{theorem}
Выясняется, что эта оценка достаточно точна.

\subsection{Вспомогательная сумма специального вида}
\begin{lemma}
	Если $p\geq 2$, то
	\begin{equation}\label{summa_drobey}
		\sum_{i=0}^{p-1} \frac{i \cdot 2^i}{p} = \frac{2^p(p-2) + 2}{p}
	\end{equation}
\end{lemma}

% В Демидовиче этого не нашёл

\paragraph{Доказательство.}
Равенство \eqref{summa_drobey} равносильно равенству
\begin{equation}\label{summa_drobey_multiplied}
	\sum_{i=0}^{p-1} i \cdot 2^i = 2^p(p-2) + 2
	.
\end{equation}
Докажем это равенство методом математической индукции.

\paragraph{База индукции.}
Для $p=2$ имеем
\begin{equation}
	\sum_{i=0}^{2-1} i \cdot 2^i = 0 \cdot 2^0 + 1 \cdot 2^1 = 2
\end{equation}
и
\begin{equation}
	2^2(2-2) + 2 = 2
	.
\end{equation}
Видим, что для $p=2$ соотношение \eqref{summa_drobey_multiplied} выполняется.

\paragraph{Шаг индукции.}
Пусть соотношение \eqref{summa_drobey_multiplied} выполняется для $p=m$, $m\geq 2$, т.е.
\begin{equation}\label{summa_drobey_multiplied_m}
	\sum_{i=0}^{m-1} i \cdot 2^i = 2^m(m-2) + 2
	.
\end{equation}

Покажем, что тогда соотношение \eqref{summa_drobey_multiplied} выполняется и для $p=m+1$.
Действительно,
\begin{multline}
	\sum_{i=0}^{(m+1)-1} i \cdot 2^i
	=
	\sum_{i=0}^{m} i \cdot 2^i
	=
	\sum_{i=0}^{m - 1} i \cdot 2^i + m\cdot 2 ^m
	\mathop{=}^{\eqref{summa_drobey_multiplied_m}}
	2^m(m-2) + 2 + m\cdot 2 ^m
	=
	\\=
	m\cdot2^m-2\cdot2^m  + 2 + m\cdot 2 ^m
	=
	m\cdot2^{m+1}-2^{m+1}  + 2
	=
	\\=
	2^{m+1}(m-1)  + 2
	=
	2^{m+1}((m+1)-1)  + 2
	,
\end{multline}
т.е. соотношение \eqref{summa_drobey_multiplied} выполняется и для $p=m+1$,
что и требовалось доказать.



\subsection{Вспомогательный оператор $S$}
Пусть $y\in \ell_\infty$.
Определим оператор $S:\ell_\infty \to \ell_\infty$ следующим образом:
\begin{equation}\label{operator_S}
	(Sy)_k = y_{i+2}, \mbox{ где } 2^i < k \leq 2^i+1
\end{equation}
Этот оператор вводится исключительно для упрощения изложения конструкции.

\begin{example}
	$$
		S(\{1,2,3,4,5,6, ...\}) = \{1,2,3,3,4,4,4,4,5,5,5,5,5,5,5,5,6...\}
	$$
\end{example}

Теперь нам потребуются некоторые свойства оператора $S$.

\begin{lemma}
	\label{thm:alpha_S}
	\begin{equation}\label{alpha_S}
		\alpha(Sx) = \varlimsup_{k\to\infty} |x_{k+1} - x_{k}|
	\end{equation}
\end{lemma}

\paragraph{Доказательство.}

\begin{equation*}
	\alpha(Sx) =
	\varlimsup_{i\to\infty} \sup_{i < j \leq 2i} | (Sx)_i - (Sx)_j | = ...
\end{equation*}
Положим для каждого $i$ число $m_i$ так,
что $m_i = 2^{k_i}$, $i \leq m_i < 2i$
(очевидно, это всегда можно сделать).
\begin{equation*}
	... =
	\varlimsup_{i\to\infty} \max \left\{
		\max_{i   < j \leq m_i} | (Sx)_i - (Sx)_j |,
		\max_{m_i < j \leq 2i } | (Sx)_i - (Sx)_j |
	\right\} =
	...
\end{equation*}
Но при $2^{k_i - 1} < i < j \leq m_i = 2^{k_i}$
имеем $(Sx)_i = (Sx)_j$, и первый модуль обращается в нуль.

\begin{equation*}
	... =
	\varlimsup_{i\to\infty}
		\max_{m_i < j \leq 2i } | (Sx)_i - (Sx)_j |
	=
	...
\end{equation*}
Но при $2^{k_i - 1} < i \leq m_i = 2^{k_i} < j \leq 2^{k_i+1}$
имеем $(Sx)_i = x_{k_i+1}$, $(Sx)_j = x_{k_i+2}$, откуда
\begin{equation}\label{alpha_S_sosedi}
	... =
	\varlimsup_{k\to\infty}
		| x_{k+1} - x_k |
\end{equation}

Лемма доказана.

\begin{lemma}
	\begin{equation}\label{summa_S_less}
		\sum_{k=2}^{2^p} (Sy)_k =
		\sum_{i=0}^{p-1} 2^i y_{i+2}
	\end{equation}
\end{lemma}

\paragraph{Доказательство.}

\begin{equation*}
	\sum_{k=2}^{2^p} (Sy)_k =
	\sum_{i=0}^{p-1} \sum_{k=2^i+1}^{2^{i+1}} (Sy)_k =
	\sum_{i=0}^{p-1} \sum_{k=2^i+1}^{2^{i+1}} y_{i+2} =
	\sum_{i=0}^{p-1} 2^i y_{i+2}
\end{equation*}

Лемма доказана.


\begin{lemma}
	\begin{equation}\label{summa_S}
		\sum_{k=2^i+1}^{2^{i+j+1}} (Sx)_k =
		2^i\sum_{k=2}^{2^{j+1}} (ST^ix)_k
	\end{equation}

	Здесь и далее $(Tx)_n = x_{n+1}$.
\end{lemma}

\paragraph{Доказательство.}

\begin{multline*}
	\sum_{k=2^i+1}^{2^{i+j+1}} (Sx)_k =
	\sum_{m = i}^{i+j}\sum_{k=2^m+1}^{2^{m+1}} (Sx)_k =
	\sum_{m = i}^{i+j}2^m \cdot x_{m+2} =
	\\=
	2^i \cdot \sum_{n = 0}^{j}2^n \cdot x_{n+2+i} =
	2^i \cdot \sum_{n = 0}^{j}2^n (T^i x)_{n+2} =
	2^i \cdot \sum_{k=2}^{2^{j+1}} (ST^i x)_k
\end{multline*}

Лемма доказана.

\subsection{Вспомогательная функция $k_b$}

Введём функцию
\begin{equation}\label{def_k_b}
	k_b(x) = \frac{1}{2b}\left|
		\sum_{k=1}^{b}x_k - \sum_{k=b+1}^{2b}x_k
	\right|
\end{equation}

\begin{lemma}
	\begin{equation}\label{alpha_greater_k_b}
		\alpha (Cx) \geq \varlimsup_{i\to \infty} k_i(x)
	\end{equation}
\end{lemma}

\paragraph{Доказательство.}

\begin{multline*}
	\alpha (Cx) \mathop{=}\limits^{def}
	\varlimsup_{i\to \infty} \sup_{i<j\leq 2i} |(Cx)_i - (Cx)_j| \geq
	\varlimsup_{i\to \infty} |(Cx)_i - (Cx)_{2i}| =
	\\ =
	\varlimsup_{i\to \infty} \left|\frac{1}{i}\sum_{k=1}^i  - \frac{1}{2i}\sum_{k=1}^{2i} \right| =
	\varlimsup_{i\to \infty} \left|\frac{1}{i}\sum_{k=1}^i  - \frac{1}{2i}\sum_{k=1}^{i}- \frac{1}{2i}\sum_{k=i+1}^{2i}\right| =
	\\=
	\varlimsup_{i\to \infty} \left|\frac{1}{2i}\sum_{k=1}^i - \frac{1}{2i}\sum_{k=i+1}^{2i}\right| =
	\varlimsup_{i\to \infty} k_i(x)
\end{multline*}

\paragraph{Примечание.}
Введение функции $k_b(x)$ позволит нам в дальнейшем перейти от работы с оператором Чезаро
к несложным преобразованиям сумм.



\subsection{Основные построения}

Построим вектор $y\in \ell_\infty$ следующим образом:

\begin{equation}\label{y_construction}
	y = \left\{
		0, 0, \frac{1}{p}, \frac{2}{p}, \frac{3}{p},
		...,
		\frac{p-1}{p}, 1, \frac{p-1}{p},
		...,
		\frac{1}{p},
		~~
		\underbrace{
		\phantom{\frac{1}{1}\!\!\!}
			0, 0, 0, ..., 0,
		}_\text{$\phantom{\frac{1}{1}\!\!\!}$\!\!\!$3p+1$ раз}
		~~
		\frac{1}{p}, ...
	\right\}
\end{equation}
так, что
\begin{equation}\label{T_y}
	T^{5p}y = y
\end{equation}
%(0 повторяется $3p+1$ раз или около того, надо будет ещё очень аккуратно пересчитать).


Положим $x = Sy$.
Тогда с учётом (\ref{alpha_S})
\begin{equation}\label{alpha_x}
	\alpha (x) = \alpha (Sy) = \frac{1}{p}
\end{equation}


Оценим $\alpha(Cx)$:

\begin{multline*}
	\alpha (Cx) \mathop{\geq}^{(\ref{alpha_greater_k_b})}
	\varlimsup_{b\to \infty} k_b(x) =
	\varlimsup_{b\to \infty}\frac{1}{2b}\left|
		\sum_{k=1}^{b}x_k - \sum_{k=b+1}^{2b}x_k
	\right| \geq
	\\ \geq
	\varlimsup_{
		i\to \infty,~
		b=2^i~
	}\frac{1}{2^{i+1}}\left|
		\sum_{k=1}^{2^i}(Sy)_k - \sum_{k=2^i+1}^{2^{i+1}}(Sy)_k
	\right| =
	\\=
	\varlimsup_{i\to \infty}\frac{1}{2^{i+1}}\left|
		\sum_{k=1}^{2^i}(Sy)_k - 2^i y_{i+2}
	\right| =
	\varlimsup_{i\to \infty}\left|
		\frac{1}{2^{i+1}}\sum_{k=1}^{2^i}(Sy)_k - \frac{y_{i+2}}{2}
	\right| \geq
\end{multline*}
\begin{multline*}
	\\ \geq
	\varlimsup_{
		m\to \infty,~
		i=5pm+p~
	}\left|
		\frac{1}{2^{5pm+p+1}}\sum_{k=1}^{2^{5pm+p}}(Sy)_k - \frac{y_{5pm+p+2}}{2}
	\right| =
	\\=
	\varlimsup_{m\to \infty}\left|
		\frac{1}{2^{5pm+p+1}}\sum_{k=1}^{2^{5pm+p}}(Sy)_k - \frac{1}{2}
	\right| =
	\\=
	\varlimsup_{m\to \infty}\left|
		\frac{1}{2^{5pm+p+1}}\sum_{k=1}^{2^{5pm}}(Sy)_k
		+
		\frac{1}{2^{5pm+p+1}}\sum_{k=2^{5pm}+1}^{2^{5pm+p}}(Sy)_k
		- \frac{1}{2}
	\right|
	\mathop{=}^{(\ref{summa_S})}
	\\=
	\varlimsup_{m\to \infty}\left|
		\frac{1}{2^{5pm+p+1}}\sum_{k=1}^{2^{5pm}}(Sy)_k
		+
		\frac{1}{2^{5pm+p+1}} \cdot 2^{5pm} \cdot \sum_{k=2}^{2^p}(ST^{5pm}y)_k
		- \frac{1}{2}
	\right|
	\mathop{=}^{(\ref{T_y})}
	\\=
	\varlimsup_{m\to \infty}\left|
		\frac{1}{2^{5pm+p+1}}\sum_{k=1}^{2^{5pm}}(Sy)_k
		+
		\frac{1}{2^{5pm+p+1}} \cdot 2^{5pm} \cdot \sum_{k=2}^{2^p}(Sy)_k
		- \frac{1}{2}
	\right| =
	\\=
	\varlimsup_{m\to \infty}\left|
		\frac{1}{2^{5pm+p+1}}\sum_{k=1}^{2^{5pm}}(Sy)_k
		+
		\frac{1}{2^{p+1}} \sum_{k=2}^{2^p}(Sy)_k
		- \frac{1}{2}
	\right|
	\mathop{=}^{(\ref{summa_S_less})}
	\\=
	\varlimsup_{m\to \infty}\left|
		\frac{1}{2^{5pm+p+1}}\sum_{k=1}^{2^{5pm}}(Sy)_k
		+
		\frac{1}{2^{p+1}} \sum_{i=0}^{p-1}2^i y_{i+2}
		- \frac{1}{2}
	\right| =
	\\=
	\varlimsup_{m\to \infty}\left|
		\frac{1}{2^{5pm+p+1}}\sum_{k=1}^{2^{5pm}}(Sy)_k
		+
		\frac{1}{2^{p+1}} \sum_{i=0}^{p-1}2^i \cdot \frac{i}{p}
		- \frac{1}{2}
	\right|
	\mathop{=}^{(\ref{summa_drobey})}
	\\=
	\varlimsup_{m\to \infty}\left|
		\frac{1}{2^{5pm+p+1}}\sum_{k=1}^{2^{5pm}}(Sy)_k
		+
		\frac{1}{2^{p+1}} \cdot \frac{2^p(p-2)+2}{p}
		- \frac{1}{2}
	\right| =
	\\=
	\varlimsup_{m\to \infty}\left|
		\frac{1}{2^{5pm+p+1}}\sum_{k=1}^{2^{5pm}}(Sy)_k
		+
		\frac{1}{2} \cdot \frac{p-2}{p} + \frac{1}{p 2^p}
		- \frac{1}{2}
	\right| =
	\\=
	\varlimsup_{m\to \infty}\left|
		\frac{1}{2^{5pm+p+1}}\sum_{k=1}^{2^{5pm}}(Sy)_k
		-
		\frac{1}{p} + \frac{1}{p 2^p}
	\right| =
	\\=
	\varlimsup_{m\to \infty}\left|
		\frac{1}{2^{5pm+p+1}}\sum_{k=1}^{2^{5pm-2p}}(Sy)_k
		+
		\frac{1}{2^{5pm+p+1}}\sum_{k=2^{5pm-2p}+1}^{2^{5pm}}(Sy)_k
		-\frac{1}{p} + \frac{1}{p 2^p}
	\right| =
	\\\mbox{но во второй сумме все $(Sy)_k$ --- нули по построению}
\end{multline*}
\begin{multline*}
	\\=
	\varlimsup_{m\to \infty}\left|
		\frac{1}{2^{5pm+p+1}}\sum_{k=1}^{2^{5pm-2p}}(Sy)_k
		-\frac{1}{p} + \frac{1}{p 2^p}
	\right| = h
\end{multline*}

Но $0 \leq (Sy)_k \leq 1$,
значит,
$$
	\frac{1}{2^{5pm+p+1}}\sum_{k=1}^{2^{5pm-2p}}(Sy)_k
	\leq
	\frac{1}{2^{5pm+p+1}} \cdot 2^{5pm-2p}
	=
	\frac{1}{2^{3p+1}}
$$
Модуль раскрываем со знаком ``-''

\begin{multline*}
	h=
	\varlimsup_{m\to \infty} \left(
		\frac{1}{p} (1-2^{-p})
		- \frac{1}{2^{3p+1}}
	\right) =
	\frac{1}{p} (1-2^{-p})
	- \frac{1}{2^{3p+1}}
	= \\ =
	\frac{1}{p} (1-2^{-p})
	- \frac{1}{2^{2p+1}} \cdot 2^{-p}
	>
	\frac{1}{p} (1-2^{-p})
	- \frac{1}{p} \cdot 2^{-p}
	=
	\frac{1}{p} (1-2^{-p+1})
\end{multline*}


Таким образом,
$$
	\frac{\alpha(Cx)}{\alpha(x)} \geq
	\frac{	\frac{1}{p} (1-2^{-p+1}) }{\frac{1}{p}} =
	1-2^{-p+1}
$$

Рассматривая $x$ как функцию от $p$, имеем:
$$
	\sup_{p\in\mathbb{N}} \frac{\alpha(Cx(p))}{\alpha(x(p))} \geq
	\sup_{p\in\mathbb{N}} (1-2^{-p+1}) =
	1
	.
$$
Таким образом, может быть сформулирована следующая

\begin{theorem}
	\label{thm:alpha_Cx_no_gamma}
	\begin{equation}
		\sup_{x\in\ell_\infty, \alpha(x)\neq 0} \frac{\alpha(Cx)}{\alpha(x)}=1
		.
	\end{equation}
\end{theorem}

\subsection{Некоторые гипотезы}

\begin{hypothesis}
	Пусть $x\in\ell_\infty$ и $0 \leq x_k \leq 1$.
	Тогда для любого $n\in\mathbb{N}$
	\begin{equation}
		\alpha(C^n x) \leq \frac{1}{2^n}
		.
	\end{equation}
\end{hypothesis}

\begin{hypothesis}
	Для любых $x\in\ell_\infty$ и $n\in\mathbb{N}$
	\begin{equation}
		\alpha(C^n x) - \alpha(C^{n+1} x) \leq \alpha(C^{n-1} x) - \alpha(C^{n} x)
		.
	\end{equation}
\end{hypothesis}

\begin{hypothesis}
	Для любого $0<a<1$ существует такой $x\in\ell_\infty$, что
	\begin{equation}
		\lim_{m\to\infty} \frac{\alpha(C^m x)}{a^m} = \infty
		.
	\end{equation}
\end{hypothesis}



\chapter{Последовательности, почти сходящиеся к нулю (пространство $ac_0$)}

	\section{Переформулировка критерия Лоренца почти сходимости последовательности к нулю}
	Дадим переформулировку критерия Лоренца почти сходимости последовательности,
которая иногда позволяет упростить доказательство.


\paragraph{Теорема.}
(Критерий почти сходимости к нулю для неотрицательной последовательности).
Пусть $x\in\ell_{\infty}$,
$\forall(i\in\mathbb{N})[x_i\geq 0]$.

$x\in ac_0$ тогда и только тогда, когда
\begin{equation}\label{crit_pos_ac0}
	\forall(A_0\in\mathbb{N})
	\exists(n_0\in\mathbb{N})
	\exists(m_0\in\mathbb{N})
	\forall(n\geq n_0)
	\forall(m\geq m_0)
	\\
	\left[
		\frac{1}{n}
		\sum_{k=m+1}^{m+n} x_k
		<
		\frac{1}{A_0}
	\right]
	.
\end{equation}

\paragraph{Доказательство.}
По теореме Лоренца $x\in ac_0$ тогда и только тогда, когда
\begin{equation}\label{Lorencz_ac0}
	\lim_{n\to\infty} \frac{1}{n} \sum_{k=m+1}^{m+n} x_k = 0
\end{equation}
равномерно по $m$.

Или, переводя на язык кванторов,
\begin{equation}\label{crit_ac0}
	\forall(A_1\in\mathbb{N})
	\exists(n_1\in\mathbb{N})
	\forall(n\geq n_0)
	\forall(m \in \mathbb{N})
	\\
	\left[
		\frac{1}{n}
		\sum_{k=m+1}^{m+n} x_k
		<
		\frac{1}{A_1}
	\right]
	.
\end{equation}
Очевидно, что из \eqref{crit_ac0} следует \eqref{crit_pos_ac0} (например, положив $m_0 = 1$),
тем самым необходимость \eqref{crit_pos_ac0} доказана.

\paragraph{Достаточность.}
Пусть выполнено \eqref{crit_pos_ac0}.
Покажем, что выполнено \eqref{crit_ac0}.
Зафиксируем $A_1$.
Положим $A_0 = 2A_1$ и отыщем $n_0$ и $m_0$ в соотвествии с \eqref{crit_pos_ac0}.
Положим $n_1 = 2A_0(n_0+m_0)\|x\|$.
Покажем, что \eqref{crit_ac0} верно для любых $n\geq n_1$, $m\in \mathbb{N}$.
Зафиксируем $n$ и рассмотрим $m$.

Пусть сначала $m\geq m_0$.
Тогда в силу того, что $n\geq n_1 = 2A_0(n_0+m_0)\|x\| > n_0$ имеем $n>n_0$.
Применим \eqref{crit_pos_ac0}:
\begin{equation}
	\frac{1}{n}
	\sum_{k=m+1}^{m+n} x_k
	<
	\frac{1}{A_0}
	=
	\frac{1}{2A_1}
	<
	\frac{1}{A_1}
	,
\end{equation}
т.е. \eqref{crit_ac0} выполнено.

Пусть теперь $m < m_0$.
Тогда
\begin{multline}
	\frac{1}{n} \sum_{k=m+1}^{m+n} x_k
	<
	\frac{1}{n} \sum_{k=1}^{m_0+n} x_k
	=
	\\=
	\frac{1}{n} \sum_{k=1}^{m_0} x_k + \frac{1}{n} \sum_{k=m_0+1}^{m_0+n} x_k
	%\leq
	\mathop{\leq}^{\mbox{~~в силу \eqref{crit_pos_ac0}~~}}
	\frac{1}{n} \sum_{k=1}^{m_0} x_k + \frac{1}{A_0}
	\leq
	\\ \leq
	\frac{1}{n} \sum_{k=1}^{m_0} \|x\| + \frac{1}{A_0}
	\leq
	\frac{1}{n_1} \sum_{k=1}^{m_0} \|x\| + \frac{1}{A_0}
	=
	\frac{1}{2A_0(n_0+m_0)\|x\|} \sum_{k=1}^{m_0} \|x\| + \frac{1}{A_0}
	=
	\\=
	\frac{m_0\|x\|}{2A_0(n_0+m_0)\|x\|} + \frac{1}{A_0}
	=
	\frac{m_0}{2A_0(n_0+m_0)} + \frac{1}{A_0}
	\leq
	\\ \leq
	\frac{m_0}{2A_0m_0} + \frac{1}{A_0}
	=
	\frac{1}{2A_0} + \frac{1}{A_0}
	=
	\frac{3}{2A_0}
	=
	\frac{3}{4A_1}
	<
	\frac{1}{A_1},
\end{multline}
т.е. \eqref{crit_ac0} тоже выполнено.

\paragraph{Примечание.}
Для знакопеременных последовательностей условие \eqref{crit_pos_ac0} является необходимым,
но неизвестно, является ли оно достаточным.

\paragraph{Примечание.}
Удобство критерия \eqref{crit_pos_ac0} в том,
что можно выбирать $m_0$ в зависимости от $A_0$.


	\section{О почти сходимости к нулю последовательности из нулей и единиц}
	\paragraph{Теорема.}
(Критерий почти сходимости к нулю для неотрицательной последовательности).
Пусть $x\in\ell_{\infty}$,
$\forall(i\in\mathbb{N})[x_i\geq 0]$.

$x\in ac_0$ тогда и только тогда, когда
\begin{equation}\label{crit_pos_ac0}
	\forall(A_0\in\mathbb{N})
	\exists(n_0\in\mathbb{N})
	\exists(m_0\in\mathbb{N})
	\forall(n\geq n_0)
	\forall(m\geq m_0)
	\\
	\left[
		\frac{1}{n}
		\sum_{k=m+1}^{m+n} x_k
		<
		\frac{1}{A_0}
	\right]
	.
\end{equation}

\paragraph{Доказательство.}
По теореме Лоренца $x\in ac_0$ тогда и только тогда, когда
\begin{equation}\label{Lorencz_ac0}
	\lim_{n\to\infty} \frac{1}{n} \sum_{k=m+1}^{m+n} x_k = 0
\end{equation}
равномерно по $m$.

Или, переводя на язык кванторов,
\begin{equation}\label{crit_ac0}
	\forall(A_1\in\mathbb{N})
	\exists(n_1\in\mathbb{N})
	\forall(n\geq n_0)
	\forall(m \in \mathbb{N})
	\\
	\left[
		\frac{1}{n}
		\sum_{k=m+1}^{m+n} x_k
		<
		\frac{1}{A_1}
	\right]
	.
\end{equation}
Очевидно, что из \eqref{crit_ac0} следует \eqref{crit_pos_ac0} (например, положив $m_0 = 1$),
тем самым необходимость \eqref{crit_pos_ac0} доказана.

\paragraph{Достаточность.}
Пусть выполнено \eqref{crit_pos_ac0}.
Покажем, что выполнено \eqref{crit_ac0}.
Зафиксируем $A_1$.
Положим $A_0 = 2A_1$ и отыщем $n_0$ и $m_0$ в соотвествии с \eqref{crit_pos_ac0}.
Положим $n_1 = 2A_0(n_0+m_0)\|x\|$.
Покажем, что \eqref{crit_ac0} верно для любых $n\geq n_1$, $m\in \mathbb{N}$.
Зафиксируем $n$ и рассмотрим $m$.

Пусть сначала $m\geq m_0$.
Тогда в силу того, что $n\geq n_1 = 2A_0(n_0+m_0)\|x\| > n_0$ имеем $n>n_0$.
Применим \eqref{crit_pos_ac0}:
\begin{equation}
	\frac{1}{n}
	\sum_{k=m+1}^{m+n} x_k
	<
	\frac{1}{A_0}
	=
	\frac{1}{2A_1}
	<
	\frac{1}{A_1}
	,
\end{equation}
т.е. \eqref{crit_ac0} выполнено.

Пусть теперь $m < m_0$.
Тогда
\begin{multline}
	\frac{1}{n} \sum_{k=m+1}^{m+n} x_k
	<
	\frac{1}{n} \sum_{k=1}^{m_0+n} x_k
	=
	\\=
	\frac{1}{n} \sum_{k=1}^{m_0} x_k + \frac{1}{n} \sum_{k=m_0+1}^{m_0+n} x_k
	%\leq
	\mathop{\leq}^{\mbox{~~в силу \eqref{crit_pos_ac0}~~}}
	\frac{1}{n} \sum_{k=1}^{m_0} x_k + \frac{1}{A_0}
	\leq
	\\ \leq
	\frac{1}{n} \sum_{k=1}^{m_0} \|x\| + \frac{1}{A_0}
	\leq
	\frac{1}{n_1} \sum_{k=1}^{m_0} \|x\| + \frac{1}{A_0}
	=
	\frac{1}{2A_0(n_0+m_0)\|x\|} \sum_{k=1}^{m_0} \|x\| + \frac{1}{A_0}
	=
	\\=
	\frac{m_0\|x\|}{2A_0(n_0+m_0)\|x\|} + \frac{1}{A_0}
	=
	\frac{m_0}{2A_0(n_0+m_0)} + \frac{1}{A_0}
	\leq
	\\ \leq
	\frac{m_0}{2A_0m_0} + \frac{1}{A_0}
	=
	\frac{1}{2A_0} + \frac{1}{A_0}
	=
	\frac{3}{2A_0}
	=
	\frac{3}{4A_1}
	<
	\frac{1}{A_1},
\end{multline}
т.е. \eqref{crit_ac0} тоже выполнено.

\paragraph{Примечание.}
Для знакопеременных последовательностей условие \eqref{crit_pos_ac0} является необходимым,
но неизвестно, является ли оно достаточным.

\paragraph{Примечание.}
Удобство критерия \eqref{crit_pos_ac0} в том,
что можно выбирать $m_0$ в зависимости от $A_0$.

\paragraph{Задача о пасьянсе из нулей и единиц.}
Пусть $n_i$~--- строго возрастающая последовательность натуральных чисел,
\begin{equation}
	M(j) = \liminf_{i\to\infty} n_{i+j} - n_i,
\end{equation}
\begin{equation}
	x_k = \left\{\begin{array}{ll}
		1, & \mbox{~если~} k = n_i
		\\
		0  & \mbox{~иначе~}
	\end{array}\right.
\end{equation}

\paragraph{Замечание.}
Так как $M(j)$ есть нижний предел последовательности натуральных чисел,
то он всегда досигается,
т.е. $M(j)\in\mathbb{N}$.

Более того, для любого $j$ существует лишь конечное количество отрезков длины $M(j)$,
содержащих более $j$ единиц,
и бесконечное количество отрезков длины $M(j)$,
содержащих ровно $j$ единиц.

Через $E_j$ будем обозначать конец последнего отрезка длины $M(j)$,
содержащего более $j$ единиц.

\paragraph{Утверждение.}
Если $x \in ac_0$, то
\begin{equation}\label{lim_M(j)/j}
	\lim_{j \to \infty} \frac{M(j)}{j} = +\infty
	.
\end{equation}

\paragraph{Доказательство.}
Очевидно, что если
\begin{equation}
	\liminf_{j \to \infty} \frac{M(j)}{j} = +\infty
	,
\end{equation}
то выполнено \eqref{lim_M(j)/j}.
Предположим противное:
\begin{equation}
	\liminf_{j \to \infty} \frac{M(j)}{j} = с < +\infty
	.
\end{equation}
Очеивдно, что в таком случае $c>0$.

По определению нижнего предела найдётся счётное множество
$J\subset\mathbb{N}$ такое, что
\begin{equation}
	\forall(j\in J)\left[c \leq \frac{M(j)}{j} \leq c+1 \right],
\end{equation}
т.е. для любого $j\in J$ существует бесконечно много отрезков длины $j\cdot(c+1)$,
на каждом из которых не менее $j$ единиц.

Т.к. $x\in ac_0$, то
\begin{equation}\label{Lorencz_ac0_epsilon}
	\forall(\varepsilon>0)
	\exists(n_0\in\mathbb{N})
	\forall(n \geq n_0)
	\forall(m\in\mathbb{N})
	\left[
		\frac{1}{n} \sum_{k=m+1}^{m+n}x_k < \varepsilon
	\right]
	.
\end{equation}

Положим $\varepsilon = 1/(c+2)$ и отыщем $n_0$.
Положим $n\in J$, $n\geq n_0$.
(Такое $n$ всегда найдётся, т.к. $J$ счётно и $J\subset\mathbb{N}$.)
Выберем $m$ так, чтобы отрезок длины $n\cdot(c+1)$,
содержащий не менее $n$ единиц,
начинался с $m+1$.
Тогда
\begin{equation}
	\frac{1}{n\cdot(c+1)}\sum_{k=m+1}^{m+n\cdot(c+1)}x_k
	\geq
	\frac{1}{n\cdot(c+1)} \cdot n
	=
	\frac{1}{c+1}
	>
	\frac{1}{c+2}
	=
	\varepsilon,
\end{equation}
что противоречит \eqref{Lorencz_ac0_epsilon}.

Полученное противоречие завершает доказательство.


\paragraph{Утверждение.}
Если
\begin{equation}\label{lim_M(j)/j_dost}
	\lim_{j \to \infty} \frac{M(j)}{j} = +\infty
	,
\end{equation}
то $x \in ac_0$.

\paragraph{Доказательство.}

По определению предела \eqref{lim_M(j)/j_dost} означает, что
\begin{equation}\label{lim_M(j)/j_ifty_def}
	\forall(C  \in\mathbb{N})
	\exists(j_0\in\mathbb{N})
	\forall(j \geq j_0)
	\left[
		\frac{M(j)}{j}>C
	\right]
	.
\end{equation}
Покажем, что выполнен критерий почти сходимости к нулю неотрицательной последовательности
\eqref{crit_pos_ac0}, т.е.
\begin{equation}
	\forall(B  \in\mathbb{N})
	\exists(n_0\in\mathbb{N})
	\exists(m_0\in\mathbb{N})
	\forall(n\geq n_0)
	\forall(m\geq m_0)
	\\
	\left[
		\frac{1}{n}
		\sum_{k=m+1}^{m+n} x_k
		<
		\frac{1}{B}
	\right]
	.
\end{equation} Действительно, зафиксируем $B$.
Используя \eqref{lim_M(j)/j_ifty_def} и положив $C=2B$,
отыщем $j_0$ такое, что для любого $j\geq j_0$ выполнено
$M(j)>2Bj$.
Положим $n_0 = 2Bj_0$.
Выберем
$$
	m_0 = 2+\max_{1\leq j \leq j_0} E_j
	.
$$

Тогда для любых $m\geq m_0$ и $n\geq n_0$ имеем
\begin{multline}
	\frac{1}{n} \sum_{k=m+1}^{m+n} x_k
	<
	\frac{1}{n} \cdot \left( \frac{n}{M(j_0)} + 1 \right) j_0
	=
	\frac{j_0}{M(j_0)} + \frac{j_0}{n}
	\leq
	\frac{1}{2B} + \frac{j_0}{n}
	\leq
	\\ \leq
	\frac{1}{2B} + \frac{j_0}{n_0}
	=
	\frac{1}{2B} + \frac{1}{2B}
	=
	\frac{1}{B}
	,
\end{multline}
т.е. условие критерия выполнено
и $x\in ac_0$,
что и требовалось доказать.

Последовательность $\{M(j)\}$, как легко выяснить, удовлетворяет некоторым условиям.


	\section{Замечание о свойствах последовательности $M(j)$}
	
Пусть $n_i$~--- строго возрастающая последовательность натуральных чисел,
\begin{equation}
	C_j = \liminf_{i\to\infty} n_{i+j} - n_i
\end{equation}

Тогда для любых $i$, $j$ имеет место быть
\begin{equation}\label{C_j_addit}
	C_{i+j} \geq C_i + C_j
	.
\end{equation}
\paragraph{Доказательство.}
По определению нижнего предела существует лишь конечное число номеров $k$
таких, что $n_{k+i} - n_k < C_i$ или $n_{k+j} - n_k < C_j$.
Зафиксируем $p$, большее всех таких $k$.

По определению нижнего предела и с учётом того, что выражение под знаком предела
принимает лишь натуральные значения,
существует бесконечное количество номеров $q$ таких, что $n_{q+i+j} - n_q = C_{i+j}$.

Обозначим через $s$ некоторый такой номер, больший $p$.
Тогда
\begin{multline}
	C_{i+j} = n_{s+i+j} - n_s = n_{s+i+j} - n_{s+i} + n_{s+i} - n_s
	\geq \\
	\geq C_j + n_{s+i} - n_s \geq C_j + C_i,
\end{multline}
так как из $s>p$ следует, что $n_{s+i+j} - n_{s+i} \geq C_j$ и $n_{s+i} - n_s \geq C_i$.

\begin{example}
	Последовательность $C_j = j+1$ не удовлетворяет условию \eqref{C_j_addit}:
	действительно, $3=C_2 < C_1+C_1 = 2+2 = 4$.
\end{example}

\paragraph{Примечание.}
Условие \eqref{C_j_addit} является необходимым, но неизвестно, является ли оно достаточным.


	\section{Срезочный критерий почти сходимости к нулю неотрицательной последовательности}
	Результаты этого пункта опубликованы в~\cite{our-mz2019ac0}.

Определим (нелинейный) оператор $\lambda$--срезки $A_\lambda$ на пространстве $\ell_\infty$.
Для $x = (x_1, x_2, ...) \in \ell_\infty$ положим
\begin{equation}
	(A_\lambda x)_k = \begin{cases}
		1, & \mbox{~если~} x_k \geq \lambda
		\\
		0  & \mbox{~иначе.~}
	\end{cases}
\end{equation}

\begin{theorem}
	[Срезочный критерий почти сходимости к нулю неотрицательной последовательности]
	Пусть $x\in\ell_\infty$, $x\geq 0$.
	Тогда
	\begin{equation}
		x\in ac_0 \Leftrightarrow
		\forall(\lambda>0)[A_\lambda x \in ac_0]
		.
	\end{equation}
\end{theorem}

\paragraph{Необходимость.}
Пусть $x\in ac_0$.
Зафиксируем $\lambda > 0$.
Пусть $y=A_\lambda x$, тогда
\begin{equation}
	(\lambda y)_k = \begin{cases}
		\lambda, & \mbox{~если~} x_k \geq \lambda
		\\
		0  & \mbox{~иначе~}
	\end{cases}
\end{equation}
Таким образом, $0 \leq \lambda y \leq x$.
Следовательно, если $x \in ac_0$,
то $\lambda y \in ac_0$ и $y \in ac_0$,
что и требовалось доказать.

\paragraph{Достаточность.}
Очевидно, что
\begin{equation}
	x\in ac_0 \Leftrightarrow
	\frac{x}{\|x\|}\in ac_0
	.
\end{equation}
Поэтому, не теряя общности, будем полагать $\|x\|\leq 1$.
Более того,
\begin{equation}
	A_\lambda x \in ac_0 \Leftrightarrow
	(1-\lambda)A_\lambda x \in ac_0
	.
\end{equation}

Предположим противное, т.е. $x\notin ac_0$,
но $\forall(\lambda>0)[(1-\lambda)A_\lambda x \in ac_0]$.
Запишем покванторное отрицание критерия Лоренца:
\begin{equation}\label{ac0_lambda_Lorencz_neg}
	\exists(\varepsilon_0 > 0)
	\forall(n_0 \in \mathbb{N})
	\exists(n > n_0)
	\exists(m \in \mathbb{N})
	\left[
		\frac{1}{n}\sum_{k=m+1}^{m+n} x_k \geq \varepsilon_0
	\right]
	.
\end{equation}
Найдём такое $\varepsilon_0$ и положим
\begin{equation}
	\varepsilon = \min\left\{ \frac{\varepsilon_0}{2}, \frac{1}{2} \right\}
	.
\end{equation}
Легко видеть, что
\begin{equation}\label{ac0_lambda_Lorencz_neg_epsilon}
	\forall(n_0 \in \mathbb{N})
	\exists(n > n_0)
	\exists(m \in \mathbb{N})
	\left[
		\frac{1}{n}\sum_{k=m+1}^{m+n} x_k > \varepsilon
	\right]
	.
\end{equation}
(знак неравенства сменился на строгий, это будет играть ключевую роль в дальнейших выкладках).

Построим последовательность
\begin{equation}
	y = \left( 1 - \frac{\varepsilon}{2} \right) A_{\varepsilon/2} x
	.
\end{equation}
Заметим, что $y\in ac_0$, т.е. по критерию Лоренца
\begin{equation}\label{ac0_lambda_Lorencz}
	\forall(\varepsilon_1 > 0)
	\exists(n_1 \in \mathbb{N})
	\forall(n' > n_1)
	\forall(m' \in \mathbb{N})
	\left[
		\frac{1}{n}\sum_{k=m+1}^{m+n} y_k < \varepsilon_1
	\right]
	.
\end{equation}

Положим в \eqref{ac0_lambda_Lorencz} $\varepsilon_1 = \varepsilon/2$
и отыщем $n_1$.
Положим в \eqref{ac0_lambda_Lorencz_neg_epsilon}
$n_0 = n_1$ и отыщем $n$ и $m$.
Положим в \eqref{ac0_lambda_Lorencz} $n' = n$, $m' = m$.
Тогда получим, что по \eqref{ac0_lambda_Lorencz_neg_epsilon}
\begin{equation}
	\frac{1}{n}\sum_{k=m+1}^{m+n} x_k > \varepsilon
	.
\end{equation}
С другой стороны, по \eqref{ac0_lambda_Lorencz}
\begin{equation}
	\frac{1}{n}\sum_{k=m+1}^{m+n} y_k < \varepsilon/2
	.
\end{equation}
Вычитая, получим
\begin{equation}
	\frac{1}{n}\sum_{k=m+1}^{m+n} (x_k - y_k) > \varepsilon/2
	.
\end{equation}
Если среднее арифметическое чисел вида $x_k - y_k$ больше $\varepsilon/2$,
то существует хотя бы один индекс $k$ такой, что $x_k - y_k > \varepsilon/2$.

Предположим, что $k$ таково, что $x_k < \varepsilon/2$.
Тогда $y_k = 0$ и $x_k - y_k < \varepsilon/2$.
Значит, предположение неверно и $x_k \geq \varepsilon/2$.
Тогда $y_k = 1-\varepsilon/2$ и с учётом $\|x\|\leq 1$ имеем
\begin{equation}
	x_k - y_k \leq 1- y_k = 1 - (1-\varepsilon/2) = \varepsilon/2
	.
\end{equation}
Следовательно, требуемого индекса $k$ не существует,
и \eqref{ac0_lambda_Lorencz_neg} не выполнено.

Полученное противоречие завершает доказательство.

Сформулируем теперь этот критерий в терминах функций $M^{(\lambda)}(j)$.

\begin{theorem}
	Пусть $x\in\ell_\infty$, $x \geq 0$, $\lambda>0$.
	Обозначим через $n^{(\lambda)}_i$ возрастающую последовательность
	индексов таких элементов $x$, что $x_k \geq \lambda$ тогда и только тогда,
	когда $k=n^{(\lambda)}_i$ для некоторого $i$.
	Обозначим
	\begin{equation}
		M^{(\lambda)}(j) = \liminf_{i\to\infty} n^{(\lambda)}_{i+j} - n^{(\lambda)}_i
		.
	\end{equation}


	Тогда для того, чтобы $x\in ac_0$, необходимо и достаточно, чтобы
	для любого $\lambda>0$ было выполнено
	\begin{equation}
		\lim_{j \to \infty} \frac{M^{(\lambda)}(j)}{j} = +\infty
		.
	\end{equation}
\end{theorem}


\end{document}

Сходящиеся последовательности, т.е. последовательности, имеющие предел в смысле классического математического анализа,
изучены достаточно хорошо.
В частности, любая сходящаяся последовательность является ограниченной.
Пространство ограниченных последовательностей будем, вслед за классиками \cite{wojtaszczyk1996banach,lindenstrauss1973classical},
обозначать через $\ell_\infty$ и снабжать его нормой
\begin{equation*}
	\|x\| = \sup_{n\in\mathbb{N}}|x_n|
	.
\end{equation*}

Однако в приложениях часто возникают ограниченные последовательности,
которые не являются сходящимися.
В таком случае возникает закономерный вопрос:
как измерить <<недостаток сходимости>>?
<<насколько не сходится>> последовательность?

Наиболее очевидным кажется вычисление расстояния $\rho(x,c)$ от заданного элемента $x\in\ell_\infty$
до пространства сходящихся последовательностей $c$
(которое равно половине разности верхнего и нижнего пределов последовательности).
Однако выясняется, что имеют место быть и другие подходы.

Нетрудно заметить, что операция взятия классического предела на пространстве сходящихся последовательностей
является непрерывным (в норме $\ell_\infty$) линейным функционалом.
Банах доказал \cite{banach1993theorie}, что этот функционал может быть непрерывно продолжен на всё пространство $\ell_\infty$.
На основе этой идеи были определены банаховы пределы
(иногда также называемые пределами Банаха--Мазура \cite{alekhno2012superposition,alekhno2015banach})
следующим образом.

Банаховым пределом называется функционал $B\in \ell_\infty^*$ такой, что:
\begin{enumerate}
	\item
		$B \geqslant 0$
	\item
		$B\mathbb{I} = 1$
	\item
		$B=BT$
\end{enumerate}
Здесь $\mathbb{I}=(1,1,1,1,1,...)$,
$T$~--- оператор сдвига: $T(x_1, x_2, x_3, ...) = (x_2, x_3, ...)$.

Простейшие свойства:
\begin{itemize}
	\item
		$\|B\|_{\ell_\infty^*} = 1$
	\item
		$Bx = \lim\limits_{n\to\infty} x_n$ для любого $x=(x_1, x_2, ...) \in c$.

		Таким образом,
		банахов предел~--- действительно естественное обобщение понятия предела сходящейся последовательности
		на все ограниченные последовательности.
\end{itemize}

Множество банаховых пределов обычно обозначают через $\mathfrak{B}$
(реже через $BM$~--- см., например, \cite{alekhno2012superposition,alekhno2015banach}).

Лоренц \cite{lorentz1948contribution} установил, что существует подпространство $\ell_\infty$,
на котором все банаховы пределы принимают одинаковое значение.
Это пространство названо пространством почти сходящихся последовательностей и обычно обозначается $ac$
(от англ. <<almost convergent>>).
Включение $c \subset ac$ собственное, т.е. $ac \setminus c \neq \varnothing$.


Обобщая критерий Лоренца, Сачестон \cite{sucheston1967banach} доказал, что для любого $x\in\ell_\infty$
и любого $B\in\mathfrak{B}$
\begin{equation*}
	q(x) =
	\lim_{n\to\infty} \sup_{m\in\mathbb{N}} \frac{1}{n} \sum_{k=m+1}^{k=m+n} x_k
	\leq
	Bx
	\leq
	\lim_{n\to\infty} \inf_{m\in\mathbb{N}} \frac{1}{n} \sum_{k=m+1}^{k=m+n} x_k
	= p(x)
\end{equation*}
и, более того,
\begin{equation*}
	\mathfrak{B}x = [q(x), p(x)]
	.
\end{equation*}

Таким образом, на вопрос: <<Насколько не сходится последовательность?>> %~---
можно давать ответ в терминах почти сходимости, т.е. принадлежности пространству $ac$,
а на вопрос: <<Насколько почти не сходится последовательность?>>~---
назвать длину отрезка $[q(x), p(x)]$.
В дальнейшем пространство почти сходящихся последовательностей неоднократно становилось предметом
различных исследований
\cite{semenov2006space,usachev2008transformations}.
В частности, в работе~\cite{connor1990almost} доказано,
что последовательность из нулей и единиц почти наверное не принадлежит пространству $ac$.
Этот факт демонстрирует, что почти сходящиеся последовательности <<достаточно редки>>.

%TODO: ссылки! Хватит или ещё?

Банаховы пределы также нашли своё применение в приложениях
\cite{semenov2015banachtraces,semenov2009fourier,strukova2015spectres}.

%TODO: ссылки! Хватит или ещё?


В настоящей работе рассматриваются некоторые вопросы асимптотических характеристик ограниченных последовательностей.

В главе 1
%TODO: \ref ???
изучается $\alpha$--функция, введённая в~\cite{our-vzms-2018}.
%
%TODO: ссылка на статью Семенова!
%
Поскольку $\alpha(c)=0$,
то $\alpha$--функцию также можно считать <<мерой несходимости>> последовательности;
равенство $\alpha(x) = 0$, однако, вовсе не гарантирует сходимость.

Устанавливается, что $\alpha$--функция не инвариантна относительно оператора сдвига $T$,
и даётся оценка на $\alpha(T^n x)$.
С другой стороны, $\alpha$--функция, в отличие от некоторых банаховых пределов
\cite{Semenov2010invariant,Semenov2011dan},
инвариантна относительно операторов растяжения $\sigma_n$.
Рассмотрены и другие свойства $\alpha$--функции.

В главе 2 обсуждается пространство $ac$ и его подпространство $ac_0$,
даётся критерий почти сходимости к нулю (т.е. принадлежности пространству $ac_0$)
знакопоcтоянной последовательности.
Затем выявляется связь между $\alpha$--функцией, расстоянием от элемента до пространства $c$
и почти сходимостью.

Наконец, в главе 3 изложены некоторые примеры операторов и найдены множества банаховых пределов,
инвариантных относительно этих операторов.

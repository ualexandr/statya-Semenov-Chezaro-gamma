\documentclass[a4paper,12pt,openbib]{report}
\usepackage{amsmath}
\usepackage[utf8]{inputenc}
\usepackage[english,russian]{babel}
\usepackage{amsfonts}
\usepackage{amsfonts,amssymb}
\usepackage{amssymb}
\usepackage{latexsym}
\usepackage{euscript}
\usepackage{enumerate}
\usepackage{graphics}
\usepackage[dvips]{graphicx}
\usepackage{geometry}
\usepackage{wrapfig}


\geometry{verbose,a4paper,tmargin=1.75cm,bmargin=2.1cm,lmargin=1.75cm,rmargin=1.75cm}

\righthyphenmin=2

\usepackage[backend=biber,style=gost-numeric,sorting=none]{biblatex}
\addbibresource{bib/Semenov.bib}
\addbibresource{bib/my.bib}
\addbibresource{bib/ext.bib}

\usepackage{etoolbox}
\patchcmd{\thebibliography}{\chapter*}{\section*}{}{}


\begin{document}

\clubpenalty=10000
\widowpenalty=10000



\begin{center}{ \bf \Large Отчёт по НИР \\ за весенний семестр 2017/18 у.г.}%
\end{center}

С целью получения и систематизации базовых знаний по функциональному анализу,
а также развития навыков работы с литературой на иностранном (английском) языке изучалась монография~\cite{wojtaszczyk1996banach}.

С целью получения специальных знаний по теме научной деятельности (банаховы пределы, инвариантные оносительно операторов)
изучалась статья~\cite{Semenov2010invariant}.

Полученные результаты опубликованы в~\cite{our-vvmsh-2018}.

\begingroup
\renewcommand{\cleardoublepage}{}
\renewcommand{\clearpage}{}
\printbibliography
\endgroup

<< \underline{~~~} >> июня 2018г.

\vspace{2em}

Магистрант \hfill \underline{~~~~~~~~~~~}\,\,\,\,Авдеев Н.Н.\newline

\vspace{2em}

Рекомендую оценку <<отлично>>

\vspace{2em}

Научный руководитель \hfill \underline{~~~~~~~~~~~}\,\,\,\,д.ф.-м.н., проф. Семенов Е.М.\newline


\end{document}

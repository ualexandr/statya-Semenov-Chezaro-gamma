\begin{theorem}
	\label{thm:alpha_sigma_n}
	Для любого $x\in\ell_\infty$ и для любого натурального $n$ верно равенство
	\begin{equation}
		\alpha(\sigma_n x) = \alpha(x)
		.
	\end{equation}
\end{theorem}

\paragraph{Доказательство.}
По определению
\begin{equation}
	\alpha(x) = \varlimsup_{i\to\infty} \max_{i<j\leqslant 2i} |x_i - x_j|
\end{equation}

Положим
\begin{equation}
	\alpha_i(x) =
	\max_{i<j\leqslant 2i} |x_i - x_j| =
	\max_{i\leqslant j\leqslant 2i} |x_i - x_j|
\end{equation}

Тогда
\begin{equation}
	\alpha(x) = \varlimsup_{i\to\infty} \alpha_i(x)
\end{equation}

Пусть $y = \sigma_n x$.
Тогда для $k=1, ..., n-1$, $a\in\mathbb{N}$ имеем
\begin{multline}
	\alpha_{an-k}(y) =
	\max_{an-k \leqslant j \leqslant 2an-2k} |y_{an-k} - y_j| =
	\\=
	(\mbox{т.к.}~y_{an-(n-1)}=y_{an-(n-2)}=...=y_{an-k}=...=y_{an-1}=y_{an})=
	\\=
	\max_{an \leqslant j \leqslant 2an-2k} |y_{an} - y_j| \leqslant
	\\ \leqslant
	(\mbox{переходим к максимуму по большему множеству}) \leqslant
	\\ \leqslant
	\max_{an \leqslant j \leqslant 2an} |y_{an} - y_j| =
	\alpha_{an}(y)
\end{multline}

С другой стороны,
\begin{multline}
	\alpha_{an}(y) =
	\max_{an \leqslant j \leqslant 2an} |y_{an} - y_j| =
	\\ =
	(\mbox{т.к.}~y=\sigma_n x,~\mbox{можем рассматривать только}~j=kn)=
	\\ =
	\max_{an \leqslant kn \leqslant 2an} |y_{an} - y_{kn}| =
	\max_{a \leqslant k \leqslant 2a} |y_{an} - y_{kn}| =
	\max_{a \leqslant k \leqslant 2a} |x_a - x_k| =
	\alpha_a(x)
\end{multline}

Таким образом, для $k=1, ..., n-1$, $a\in\mathbb{N}$ имеем соотношения:
\begin{gather}
	\alpha_{an}(y) = \alpha_a(x),
\\
	\alpha_{an-k}(y) \leqslant \alpha_a(x),
\end{gather}
откуда немедленно следует, что
\begin{equation}
	\varlimsup_{i\to\infty} \alpha_i(y) =
	\varlimsup_{i\to\infty} \alpha_i(x),
\end{equation}
т.е.
\begin{equation}
	\alpha(\sigma_n x) = \alpha(x),
\end{equation}
что и требовалось доказать.

В статье \cite[lemma 16]{Semenov2010invariant} доказывается, что
\begin{equation}
	\sigma_2 C - C \sigma_2 : \ell_\infty \to c_0
	.
\end{equation}

\begin{corollary}
	$$
		\alpha(C\sigma_2 x) =
		\alpha(\sigma_2 Cx) =
		\alpha(Cx)
	$$
\end{corollary}

\begin{hypothesis}
	Для любого $n\in\mathbb{N}$
	$$
		\alpha(C\sigma_n x) =
		\alpha(\sigma_n Cx) =
		\alpha(Cx)
	$$
\end{hypothesis}

Определим на пространстве $\ell_\infty$ функционал
\begin{equation}
	F_{i,j} (x) =\frac{1}{i} \sum_{k=j+1}^{j+i} x_k
	.
\end{equation}

Можно легко доказать следующее утверждение:

\begin{lemma}
	Для любых $x\in\ell_\infty$, $B\in\mathfrak{B}$ и $\varepsilon>0$ существуют такие $i$ и $j$,
	что
	\begin{equation}
		\label{eq:sucheston_approx_epsilon}
		|Bx - F_{i,j}(x)| < \varepsilon
		.
	\end{equation}
\end{lemma}

TODO: А надо ли это доказывать в свете того, что написано ниже?

Однако, как выясняется, можно найти такие две последовательности $x$ и $y$,
банахов предел $B$ и число $\varepsilon>0$, что <<общих>> индексов в предыдущей лемме найти не удастся.

\begin{example}
	Пусть
	\begin{equation}
		x_k = \begin{cases}
			1, \mbox{~если~} 2^{2m} \leq k < 2^{2m} + m, m > 6
			\\
			0 \mbox{~иначе,~}
		\end{cases}
	\end{equation}
	\begin{equation}
		y_k = \begin{cases}
			1, \mbox{~если~} 2^{2m+1} \leq k < 2^{2m+1} + m, m > 6
			\\
			0 \mbox{~иначе.}
		\end{cases}
	\end{equation}
	Пусть $L = B|_{ac}$, где $B\in\mathfrak{B}$ ($L$, очевидно, не зависит от выбора $B$).
	Непосредственно используя конструкции из доказательства теоремы Сачестона~\cite{sucheston1967banach},
	можно последовательно продлить $L$ сначала на $A_1 = \operatorname{Lin}(ac, x)$
	так, что $L(x) = 1 = p(x)$,
	а затем и на $A_2 = \operatorname{Lin}(ac, x,y)$
	так, что $L(y) = 1 = p(y)$
	(это возможно, так как $y\notin A_1$).
	Продлив $L$ на всё $\ell_\infty$ с сохранением неравенства $Lx\leq p(x)$,
	мы получим банахов предел.
	(Здесь мы обозначаем все продолжения функционала и полученный в результате банахов предел одной и той же буквой
	для краткости.)

	Пусть теперь $\varepsilon = 1/2$.
	Попробуем найти общие $i$ и $j$.
	Для $x$ условие~\eqref{eq:sucheston_approx_epsilon} принимает вид
	\begin{equation}
		|1 - F_{i,j}(x)| < 1/2
		,
	\end{equation}
	или, с учётом неравенства $0 \leq F_{i,j}(x) \leq 1$,
	\begin{equation}
		\label{eq:suchecton_approx_x_1_2}
		F_{i,j}(x) > 1/2
		.
	\end{equation}
	Для выполнения условия~\eqref{eq:suchecton_approx_x_1_2} необходимо (хотя и не достаточно),
	чтобы выполнялись неравенства
	\begin{equation}
		\label{eq:sucheston_approx_cases_j_x}
		\begin{cases}
			 2^{2m} - m \leq j < 2^{2m} + m,
			 \\
			 i \leq 2m
			 .
		\end{cases}
	\end{equation}
	Аналогично для $y$ получаем условия
	\begin{equation}
		\label{eq:sucheston_approx_cases_j_y}
		\begin{cases}
			 2^{2n+1} - n \leq j < 2^{2n+1} + n,
			 \\
			 i \leq 2n
			 .
		\end{cases}
	\end{equation}
	Легко видеть, что требования, накладываемые на $j$ в~\eqref{eq:sucheston_approx_cases_j_x} и~\eqref{eq:sucheston_approx_cases_j_y},
	несовместимы.
	Следовательно, <<общих>> индексов действительно не существует.
\end{example}

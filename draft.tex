\documentclass[a4paper,14pt]{article} %размер бумаги устанавливаем А4, шрифт 12пунктов
\usepackage[T2A]{fontenc}
\usepackage[utf8]{inputenc}
\usepackage[english,russian]{babel} %используем русский и английский языки с переносами
\usepackage{amssymb,amsfonts,amsmath,mathtext,cite,enumerate,float,amsthm} %подключаем нужные пакеты расширений
\usepackage[unicode,colorlinks=true,citecolor=black,linkcolor=black]{hyperref}
%\usepackage[pdftex,unicode,colorlinks=true,linkcolor=blue]{hyperref}
\usepackage{indentfirst} % включить отступ у первого абзаца
\usepackage[dvips]{graphicx} %хотим вставлять рисунки?
\graphicspath{{illustr/}}%путь к рисункам

\makeatletter
\renewcommand{\@biblabel}[1]{#1.} % Заменяем библиографию с квадратных скобок на точку:
\makeatother %Смысл этих трёх строчек мне непонятен, но поверим "Запискам дебианщика"

\usepackage{geometry} % Меняем поля страницы.
\geometry{left=4cm}% левое поле
\geometry{right=1cm}% правое поле
\geometry{top=2cm}% верхнее поле
\geometry{bottom=2cm}% нижнее поле

\renewcommand{\theenumi}{\arabic{enumi}}% Меняем везде перечисления на цифра.цифра
\renewcommand{\labelenumi}{\arabic{enumi}}% Меняем везде перечисления на цифра.цифра
\renewcommand{\theenumii}{.\arabic{enumii}}% Меняем везде перечисления на цифра.цифра
\renewcommand{\labelenumii}{\arabic{enumi}.\arabic{enumii}.}% Меняем везде перечисления на цифра.цифра
\renewcommand{\theenumiii}{.\arabic{enumiii}}% Меняем везде перечисления на цифра.цифра
\renewcommand{\labelenumiii}{\arabic{enumi}.\arabic{enumii}.\arabic{enumiii}.}% Меняем везде перечисления на цифра.цифра

\sloppy


\renewcommand\normalsize{\fontsize{14}{25.2pt}\selectfont}

\begin{document}
% !!!
% Здесь начинается реальный ТеХ-код
% Всё, что выше - беллетристика

Дальнейшее излагается неакадемическим (а местами - прямо-таки антиакадемическим языком)
с целью упрощения как написания, так и чтения текста.
Кроме того, в тексте много пометок и комментариев.

\section{Вспомогательные построения}

\subsection{}
\begin{equation}\label{summa_drobey}
	\sum_{i=0}^{p-1} \frac{i \cdot 2^i}{p} = \frac{2^p(p-2) + 2}{p}
\end{equation}

Как это посчитать вручную --- не знаю.
Мне это считал компьютер.
В принципе, вполне доказывавается индукцией,
но, скорее всего, тут можно поставить ссылку на какой-нибудь уважаемый задачник типа Демидовича.
Или справочник по суммам и рядам.


\subsection{Вспомогательный оператор $S$}
Пусть $y\in l_\infty$.
Определим оператор $S:l_\infty \to l_\infty$ следующим образом:
\begin{equation}\label{operator_S}
	(Sy)_k = y_{i+2}, \mbox{ где } 2^i < k \leq 2^i+1
\end{equation}
Этот оператор вводится исключительно для упрощения изложения конструкции
(а она достаточно изощрённая).
Никакая высокая тёмная магия типа теоремы Хана-Банаха не используется.
Даже линейность не используется.

\paragraph{Пример.}
$$
	S(\{1,2,3,4,5,6, ...\}) = \{1,2,3,3,4,4,4,4,5,5,5,5,5,5,5,5,6...\}
$$




\end{document}


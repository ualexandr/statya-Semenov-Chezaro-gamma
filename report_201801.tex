\documentclass[a4paper,12pt,openbib]{report}
\usepackage{amsmath}
\usepackage[utf8]{inputenc}
\usepackage[english,russian]{babel}
\usepackage{amsfonts}
\usepackage{amsfonts,amssymb}
\usepackage{amssymb}
\usepackage{latexsym}
\usepackage{euscript}
\usepackage{enumerate}
\usepackage{graphics}
\usepackage[dvips]{graphicx}
\usepackage{geometry}
\usepackage{wrapfig}
\usepackage{cite}

\usepackage{etoolbox}
\patchcmd{\thebibliography}{\chapter*}{\section*}{}{}

\geometry{verbose,a4paper,tmargin=1.75cm,bmargin=2.1cm,lmargin=1.75cm,rmargin=1.75cm}

\righthyphenmin=2


\begin{document}

\clubpenalty=10000
\widowpenalty=10000



\begin{center}{ \bf \Large Отчёт}%
\end{center}

С целью получения и систематизации базовых знаний по функциональному анализу,
а также развития навыков работы с литературой на иностранном (английском) языке изучалась монография~\cite{banachspaces}.

С целью получения специальных знаний по теме научной деятельности (асимпотические свойства оператора Чезаро)
изучалась статья~\cite{JFA}.

Полученные результаты опубликованы в~\cite{VZMS2018}.

%%%%  ОФОРМЛЕНИЕ СПИСКА ЛИТЕРАТУРЫ %%%
%\smallskip \centerline{\bf Литература}\nopagebreak
\begin{thebibliography}{99}

	\bibitem{banachspaces}
		\textit{Wojtaszczyk P.}
		Banach spaces for analysts. – Cambridge University Press, 1996. – Т. 25.

	\bibitem{JFA}
		{\it Semenov E. M., Sukochev F. A.}
		Invariant Banach limits and applications // Journal of Functional Analysis. – 2010. – Т. 259. – №. 6. – С. 1517-1541.
	 
	\bibitem{VZMS2018}
		\textit{Авдеев Н.Н., Семенов Е.М.}
		Об асимптотических свойствах оператора Чезаро
		// 
		Материалы Международной конференции «Воронежская зимняя математическая школа С. Г. Крейна – 2018»
		/ под ред. В. А. Костина. – Воронеж : Издательско"=полиграфический центр «Научная книга», 2018. – 396 с.
		ISBN 978-5-4446-1094-7
		
		
\end{thebibliography}

<<26>> января 2018г.

\vspace{2em}

Магистрант \hfill \underline{~~~~~~~~~~~}\,\,\,\,Авдеев Н.Н.\newline

\vspace{2em}

Рекомендую оценку <<отлично>>

\vspace{2em}

Научный руководитель \hfill \underline{~~~~~~~~~~~}\,\,\,\,д.ф.-м.н., проф. Семенов Е.М.\newline


\end{document}

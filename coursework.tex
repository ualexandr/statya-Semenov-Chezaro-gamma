\documentclass[a4paper,14pt]{article} %размер бумаги устанавливаем А4, шрифт 12пунктов
\usepackage[T2A]{fontenc}
\usepackage[utf8]{inputenc}
\usepackage[english,russian]{babel} %используем русский и английский языки с переносами
\usepackage{amssymb,amsfonts,amsmath,mathtext,enumerate,float,amsthm} %подключаем нужные пакеты расширений
\usepackage[unicode,colorlinks=true,citecolor=black,linkcolor=black]{hyperref}
%\usepackage[pdftex,unicode,colorlinks=true,linkcolor=blue]{hyperref}
\usepackage{indentfirst} % включить отступ у первого абзаца
\usepackage[dvips]{graphicx} %хотим вставлять рисунки?
\graphicspath{{illustr/}}%путь к рисункам

\makeatletter
\renewcommand{\@biblabel}[1]{#1.} % Заменяем библиографию с квадратных скобок на точку:
\makeatother %Смысл этих трёх строчек мне непонятен, но поверим "Запискам дебианщика"

\usepackage{geometry} % Меняем поля страницы.
\geometry{left=2cm}% левое поле
\geometry{right=1cm}% правое поле
\geometry{top=2cm}% верхнее поле
\geometry{bottom=2cm}% нижнее поле

\renewcommand{\theenumi}{\arabic{enumi}}% Меняем везде перечисления на цифра.цифра
\renewcommand{\labelenumi}{\arabic{enumi}}% Меняем везде перечисления на цифра.цифра
\renewcommand{\theenumii}{.\arabic{enumii}}% Меняем везде перечисления на цифра.цифра
\renewcommand{\labelenumii}{\arabic{enumi}.\arabic{enumii}.}% Меняем везде перечисления на цифра.цифра
\renewcommand{\theenumiii}{.\arabic{enumiii}}% Меняем везде перечисления на цифра.цифра
\renewcommand{\labelenumiii}{\arabic{enumi}.\arabic{enumii}.\arabic{enumiii}.}% Меняем везде перечисления на цифра.цифра

\sloppy


\renewcommand\normalsize{\fontsize{14}{25.2pt}\selectfont}

\usepackage[backend=biber,style=gost-numeric,sorting=none]{biblatex}
\addbibresource{bib/Semenov.bib}
\addbibresource{bib/my.bib}
\addbibresource{bib/ext.bib}

\begin{document}
% !!!
% Здесь начинается реальный ТеХ-код
% Всё, что выше - беллетристика

\setcounter{page}{2}

\tableofcontents{}
\newpage

\section{Введение}
При изучении пространства ограниченных последовательностей
$l_{\infty}$
особый интерес представляют асимпотические свойства его элементов \cite{Semenov2010invariant}.

В данной работе собраны расширенные версии работ
\cite{our-vzms-2018}
и
\cite{our-vvmsh-2018},
а также приведены некоторые другие факты.



\section{Отсутствие строго сжимающей оценки на $\alpha(Cx)$}

На пространстве ограниченных последовательностей $l_\infty$ определяется оператор Чезаро $C$
равенством
$
	(Cx)_n = {1}/{n} \cdot \sum_{k=1}^n x_k.
$
Определим на $l_\infty$ $\alpha$--функцию,
характеризующую асимптотические свойства последовательности,
равенством
$$
	\alpha(x) = \varlimsup_{i\to\infty}\sup_{i < j \leqslant 2i} |x_i - x_j|.
$$
Пусть $A = \{x\in l_\infty | 0 \leqslant x_n \leqslant 1\}$.
Асимптотические свойства оператора Чезаро удобнее сначала изучать на множестве $A$,
а затем перенормировкой распространять на всё $l_\infty$.
В \cite{Semenov2010invariant} изучается ряд свойств оператора Чезаро.
Можно легко доказать, что для $x\in A$ выполнено соотношение $\alpha(Cx) \leqslant 1/2$
и $\alpha(Cx) \leqslant \alpha(x)$.
Возникает естественный вопрос о справедливости более точной оценки.

\textbf{Теорема~1.}
{\it
	Не существует такого $\gamma < 1$,
	что для любого $x\in A$ выполнена мультипликативная оценка
	$$
		\alpha(Cx) \leqslant \gamma \cdot \alpha(x)
	,
	$$
или, что то же самое, не существует такого $p\in \mathbb{N}$,
	что для любого $x\in A$ выполнено неравенство
	$
		\alpha(Cx) \leqslant (1-2^{-p+1})\cdot \alpha(x).
	$
}

Для доказательства теоремы~1 потребуются вспомогательные построения.
\begin{equation}\label{summa_drobey}
	\sum_{i=0}^{p-1} \frac{i \cdot 2^i}{p} = \frac{2^p(p-2) + 2}{p}
\end{equation}
Введём вспомогательный оператор $S:l_\infty \to l_\infty$:
\begin{equation*}\label{operator_S}
	(Sy)_k = y_{i+2}, \mbox{ где } 2^i < k \leqslant 2^i+1
\end{equation*}
Нам потребуются следующие свойства оператора $S$.
\begin{equation}\label{alpha_S}
	\alpha(Sx) = \varlimsup_{k\to\infty} |x_{k+1} - x_{k}|
\end{equation}
\begin{equation}\label{summa_S_less}
	\sum_{k=2}^{2^p} (Sy)_k =
	\sum_{i=0}^{p-1} 2^i y_{i+2}
\end{equation}
Здесь и далее $(Tx)_n = x_{n+1}$.
\begin{equation}\label{summa_S}
	\sum_{k=2^i+1}^{2^{i+j+1}} (Sx)_k =
	2^i\sum_{k=2}^{2^{j+1}} (ST^ix)_k
\end{equation}
Введём вспомогательную функцию
%\vspace{-2.28em}
\begin{equation*}\label{def_k_b}
	k_b(x) = (2b)^{-1} \left|
		\sum\nolimits_{k=1}^{b}x_k - \sum\nolimits_{k=b+1}^{2b}x_k
	\right|
\end{equation*}
Тогда
\begin{equation}\label{alpha_greater_k_b}
	\alpha (Cx) \geqslant \varlimsup_{i\to \infty} k_i(x)
\end{equation}

\textbf{Схема доказательства теоремы~1.}
Зафиксируем $p$ и построим $y\in l_\infty$:
\begin{equation*}\label{y_construction}
	y = \left\{
		0, 0, \frac{1}{p}, \frac{2}{p}, %\frac{3}{p},
		...,
		\frac{p-1}{p}, 1, \frac{p-1}{p},
		...,
		\frac{1}{p},
		0, ..., 0,
		\frac{1}{p}, ...
	\right\}
\end{equation*}
так, что
\begin{equation}\label{T_y}
	T^{5p}y = y
\end{equation}
Положим $x = Sy$, тогда с учётом (\ref{alpha_S})
$
	\alpha (x) = \alpha (Sy) = \frac{1}{p}
$.

Оценим $\alpha(Cx)$, принимая во внимание (\ref{summa_drobey}) и (\ref{summa_S_less}) --- (\ref{alpha_greater_k_b}):
\begin{multline*}
	\alpha (Cx) \mathop{\geqslant}^{(\ref{alpha_greater_k_b})}
	\varlimsup_{b\to \infty} k_b(x) \geqslant
	\varlimsup_{
		i\to \infty,~
		b=2^i~
	}\frac{1}{2^{i+1}}\left|
		\sum_{k=1}^{2^i}(Sy)_k - \sum_{k=2^i+1}^{2^{i+1}}(Sy)_k
	\right| \geqslant
	\\ \geqslant
	\varlimsup_{
		m\to \infty,~
		i=5pm+p~
	}\left|
		\frac{1}{2^{5pm+p+1}}\sum_{k=1}^{2^{5pm+p}}(Sy)_k - \frac{y_{5pm+p+2}}{2}
	\right| =
	\\=
	\varlimsup_{m\to \infty}\left|
		\frac{1}{2^{5pm+p+1}}\sum_{k=1}^{2^{5pm}}(Sy)_k
		+
		\frac{1}{2^{5pm+p+1}}\sum_{k=2^{5pm}+1}^{2^{5pm+p}}(Sy)_k
		- \frac{1}{2}
	\right|
	\mathop{=}^{(\ref{summa_S})}
	\\=
	\varlimsup_{m\to \infty}\left|
		\frac{1}{2^{5pm+p+1}}\sum_{k=1}^{2^{5pm}}(Sy)_k
		+
		\frac{2^{5pm}}{2^{5pm+p+1}} \sum_{k=2}^{2^p}(ST^{5pm}y)_k
		- \frac{1}{2}
	\right|
	\mathop{=}^{(\ref{T_y})}
	\\=
	\varlimsup_{m\to \infty}\left|
		\frac{1}{2^{5pm+p+1}}\sum_{k=1}^{2^{5pm}}(Sy)_k
		+
		\frac{1}{2^{p+1}} \sum_{k=2}^{2^p}(Sy)_k
		- \frac{1}{2}
	\right|
	\mathop{=}^{(\ref{summa_S_less})}
	\\=
	\varlimsup_{m\to \infty}\left|
		\frac{1}{2^{5pm+p+1}}\sum_{k=1}^{2^{5pm}}(Sy)_k
		+
		\frac{1}{2^{p+1}} \sum_{i=0}^{p-1}2^i \cdot \frac{i}{p}
		- \frac{1}{2}
	\right|
	\mathop{=}^{(\ref{summa_drobey})}
%\end{multline*}
%\begin{multline*}
	\\=
	\varlimsup_{m\to \infty}\left|
		\frac{1}{2^{5pm+p+1}}\sum_{k=1}^{2^{5pm-2p}}(Sy)_k
		-\frac{1}{p} + \frac{1}{p 2^p}
	\right| \geqslant
	\\ \geqslant
	\varlimsup_{m\to \infty} \left(
		\frac{1}{p} (1-2^{-p})
		- \frac{1}{2^{3p+1}}
	\right) >
	\frac{1}{p} (1-2^{-p+1})
\end{multline*}


Таким образом,
$
	\alpha(Cx) >
	(1-2^{-p+1}) \cdot \alpha(x)
$.


\section{Оператор с конечномерным ядром, для которого не существует инвариантного банахова предела}
В работе \cite{Semenov2010invariant} изучаются условия,
при которых для оператора $H:\ell_\infty\to \ell_\infty$ существует банаховы пределы,
инваринатные относительно данного оператора, то есть такие $B\in\mathfrak{B}$,
что $B(Hx) = Bx$ дл любого $x\in\ell_\infty$,
а также приводятся примеры операторов, дл которых инвариантные банаховы пределы существуют: операторы $\sigma_n$ и оператор Чезаро $C$.

Заметим, что операторы $\sigma_n$ и $C$ имеют вырожденное ядро,
однако не для любого оператора с вырожденным ядром существует инвариантный банахов предел.

\textbf{Пример.}
Пусть для $x = (x_1, x_2, ..., x_n, ...)\in \ell_\infty$
\begin{equation*}
	Ax = (x_1, 0, x_2, 0, x_3, 0, x_4, 0, ...).
\end{equation*}
Очевидно, что $\ker A = \{0\}$.

Пусть $B\in\mathfrak{B}$, $BA = B$.
Очевидно, что
\begin{equation*}
	\frac{n-1}{2}\leqslant \sum_{k=m+1}^{m+n} (A\mathbb{I})_k \leqslant \frac{n+1}{2},
\end{equation*}
где $\mathbb{I} = (1, 1, 1, 1, 1, 1, ...)$.
Тогда по теореме Лоренца
\begin{equation*}
	BA\mathbb{I} =
	\lim_{n\to\infty} \frac{1}{n}\sum_{k=m+1}^{m+n} (A\mathbb{I})_k = \frac{1}{2}
\end{equation*}
Однако по определению банахова предела
\begin{equation*}
	B\mathbb{I} = 1 \neq \frac{1}{2} = BA\mathbb{I}.
\end{equation*}
Пришли к противоречию, следовательно, банаховых пределов, инвариантных относительно $A$, не существует.



\section{Инвариантность $\alpha$--функции относительно оператора $\sigma_n$}

\paragraph{Теорема.}
$$
	\forall(x\in l_\infty) \forall(n\in\mathbb{N})
	[
		\alpha(\sigma_n x) = \alpha(x)
	]
$$

\paragraph{Доказательство.}
По определению
\begin{equation}
	\alpha(x) = \varlimsup_{i\to\infty} \max_{i<j\leqslant 2i} |x_i - x_j|
\end{equation}

Положим
\begin{equation}
	\alpha_i(x) =
	\max_{i<j\leqslant 2i} |x_i - x_j| =
	\max_{i\leqslant j\leqslant 2i} |x_i - x_j|
\end{equation}

Тогда
\begin{equation}
	\alpha(x) = \varlimsup_{i\to\infty} \alpha_i(x)
\end{equation}

Пусть $y = \sigma_n x$.
Тогда для $k=1, ..., n-1$, $a\in\mathbb{N}$ имеем
\begin{multline}
	\alpha_{an-k}(y) =
	\max_{an-k \leqslant j \leqslant 2an-2k} |y_{an-k} - y_j| =
	\\=
	(\mbox{т.к.}~y_{an-(n-1)}=y_{an-(n-2)}=...=y_{an-k}=...=y_{an-1}=y_{an})=
	\\=
	\max_{an \leqslant j \leqslant 2an-2k} |y_{an} - y_j| \leqslant
	\\ \leqslant
	(\mbox{переходим к максимуму по большему множеству}) \leqslant
	\\ \leqslant
	\max_{an \leqslant j \leqslant 2an} |y_{an} - y_j| =
	\alpha_{an}(y)
\end{multline}

С другой стороны,
\begin{multline}
	\alpha_{an}(y) =
	\max_{an \leqslant j \leqslant 2an} |y_{an} - y_j| =
	\\ =
	(\mbox{т.к.}~y=\sigma_n x,~\mbox{можем рассматривать только}~j=kn)=
	\\ =
	\max_{an \leqslant kn \leqslant 2an} |y_{an} - y_{kn}| =
	\max_{a \leqslant k \leqslant 2a} |y_{an} - y_{kn}| =
	\max_{a \leqslant k \leqslant 2a} |x_a - x_k| =
	\alpha_a(x)
\end{multline}

Таким образом, для $k=1, ..., n-1$, $a\in\mathbb{N}$ имеем соотношения:
\begin{gather}
	\alpha_{an}(y) = \alpha_a(x),
\\
	\alpha_{an-k}(y) \leqslant \alpha_a(x),
\end{gather}
откуда немедленно следует, что
\begin{equation}
	\varlimsup_{i\to\infty} \alpha_i(y) =
	\varlimsup_{i\to\infty} \alpha_i(x),
\end{equation}
т.е.
\begin{equation}
	\alpha(\sigma_n x) = \alpha(x),
\end{equation}
что и требовалось доказать.

\paragraph{Следствие.}
$$
	\alpha(C\sigma_2 x) =
	\alpha(\sigma_2 Cx) =
	\alpha(Cx)
$$


\printbibliography{}
\end{document}

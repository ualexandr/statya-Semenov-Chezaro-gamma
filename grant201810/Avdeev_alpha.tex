\documentclass[a4paper,openbib]{report}
\usepackage{amsmath}
\usepackage[utf8]{inputenc}
\usepackage[english,russian]{babel}
\usepackage{amsfonts,amssymb}
\usepackage{latexsym}
\usepackage{euscript}
\usepackage{enumerate}
\usepackage{graphics}
\usepackage[dvips]{graphicx}
\usepackage{geometry}
\usepackage{wrapfig}
\usepackage[colorlinks=true,allcolors=black]{hyperref}



\righthyphenmin=2

\usepackage[14pt]{extsizes}

\geometry{left=3cm}% левое поле
\geometry{right=1cm}% правое поле
\geometry{top=2cm}% верхнее поле
\geometry{bottom=2cm}% нижнее поле

\renewcommand{\baselinestretch}{1.3}

\renewcommand{\leq}{\leqslant}
\renewcommand{\geq}{\geqslant} % И делись оно всё нулём!

\newcommand{\longcomment}[1]{}

\usepackage[backend=biber,style=gost-numeric,sorting=none]{biblatex}
\addbibresource{../bib/Semenov.bib}
\addbibresource{../bib/my.bib}
\addbibresource{../bib/ext.bib}

\input{../bib/ext.hyphens.bib}

\usepackage{amsthm}
\theoremstyle{definition}
\newtheorem{lemma}{Лемма}[section]
\newtheorem{theorem}[lemma]{Теорема}
\newtheorem{example}[lemma]{Пример}
\newtheorem{property}[lemma]{Свойство}
\newtheorem{corollary}{Следствие}[lemma]

%Only referenced equations are numbered
\usepackage{mathtools}
\mathtoolsset{showonlyrefs}

%\mathtoolsset{showonlyrefs=false}
% (an equation/multline to be force-numbered)
%\mathtoolsset{showonlyrefs=true}


\begin{document}
\clubpenalty=10000
\widowpenalty=10000




На пространстве ограниченных последовательностей $\ell_\infty$ с обычной нормой
\begin{equation}
	\|x\| = \sup_{n\in\mathbb{N}} |x_n|
\end{equation}
определяется функция $\alpha(x)$ следующим равенством:
\begin{equation}
	\alpha(x) = \limsup_{i\to\infty} \max_{i<j\leqslant 2i} |x_i - x_j|
	.
\end{equation}

Функция $\alpha(x)$ служит одним из способов оценить,
<<насколько не сходится>> последовательность
(наряду с вычислением расстояния до пространства сходящихcя последовательностей $c$
и с применением банаховых пределов).

В~\cite{our-vzms-2018} исследована суперпозиция функции $\alpha(x)$ и классического оператора Чезаро~\cite{Semenov2010invariant}
и даны некоторые оценки.
Планируется:
\begin{itemize}
\item
исследовать суперпозиции функции $\alpha(x)$ с классическими операторами
(операторы сдвига $T^{\pm n}$, растяжения $\sigma_n$, сжатия $\sigma_{1/n}$, оператора Чезаро $C$ и др.)
и оценить $\alpha(Ax)$ через $\alpha(x)$, где $A$~--- оператор из вышеперечисленных.
\item
исследовать множества
$\{\alpha(x) = \alpha(Tx)\}$,
$\{\alpha(x) = 0\}$
и, возможно, другие, определяемые через функцию $\alpha$.
\item
исследовать связь функции $\alpha(x)$ с почти сходимостью~\cite{lorentz1948contribution,usachev2008transformations}.

\end{itemize}


\addcontentsline{toc}{chapter}{Список литературы}
\printbibliography{}

\end{document}

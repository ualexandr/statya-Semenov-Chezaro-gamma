Пусть на множестве $\Omega=\{0,1\}^\mathbb{N}$ задана вероятностная мера <<честной монетки>> $\mu$.
Тогда, согласно~\cite{connor1990almost}, $\mu(\Omega\cap ac)=0$.
Применим этот результат к нахождению меры множества $W$,
введённого в~\cite[\S 5]{Semenov2014geomprops}.

Пусть $W$~--- множество всех последовательностей $\chi_e$, где $e =\bigcup_{k=1}^{\infty} [n_{2k-1}, n_{2k} )$
и $\{n_k \}_{k=1}^{\infty}$
удовлетворяет условию
\begin{equation}
	\label{eq:lim_j_n_kj_measure}
	\lim_{j\to\infty}\frac{n_{k+j} - n_k}{j} = \infty
\end{equation}
равномерно по $k \in \mathbb{N}$.

Вероятностная мера <<честной монетки>> означает,
что каждая последовательность из нулей и единиц соответствует бесконечной серии
бросков честной монетки, причём выпадение орла означает нуль, а выпадение решки~--- единицу.
(В практических целях рекомендуем читателю использовать рубль, поскольку он падает быстрее и охотнее.)
%TODO: выпилить!!!!
Тогда мера подмножества $\omega\subset\Omega$
равна вероятности события <<выпала одна из серий монетки, закодированных в $\omega$>>.
Например, $\mu(\Omega)=1$, $\mu(\{x\in\Omega:x_1=1, x_2=0\})=1/4$.

Введём теперь нелинейную биекцию $Q:\Omega\leftrightarrow\Omega$ по следующему правилу:
\begin{equation}
	(Qx)_k = \begin{cases}
		x_k, &\mbox{если~} k = 1,
		\\
		|x_k-x_{k-1}|&\mbox{иначе}.
	\end{cases}
\end{equation}

\begin{lemma}
	Биекция $Q$ сохраняет меру множества.
\end{lemma}
\begin{proof}
	Пусть последовательность $x$ соответствует серии бросков монетки так, как описано выше.
	Будем теперь интерпретировать ту же самую серию бросков иначе.
	Первый бросок интерпретируем так же,
	а начиная со второго сопоставим нулю событие <<выпала та же сторона монетки, что и в прошлый раз>>,
	а единице~--- противоположное событие.
	Cобытия <<выпала решка>> и <<выпала та же сторона монетки, что и в прошлый раз>> независимы
	и вероятность каждого из них равна $1/2$.
	Осталось заметить, что новая интерпретация той же серии бросков дала нам последовательность $Qx$.
\end{proof}

NB: это скользкий момент.
Не всякая биекция сохраняет меру,
но тут вроде как спасает то, что мы интерпретируем ту же самую последовательность бросков двумя разными способами.

Заметим теперь, что из равномерного стремления~\ref{eq:lim_j_n_kj_measure}
следует, что
\begin{equation}
	\lim_{j\to\infty} \liminf_{k\to\infty} \frac{n_{k+j} - n_k}{j} = \infty
	,
\end{equation}

TODO: очевиден ли переход выше или расписать в кванторах?

откуда по лемме~\ref{thm:lim_M(j)/j_dost} получаем $QW \subset \Omega\cap ac_0$.
Тогда
\begin{equation}
	0 \leq \mu(W) = \mu(QW) \leq \mu(\Omega\cap ac_0) \leq \mu(\Omega\cap ac) = 0
	,
\end{equation}
откуда $\mu(W)=0$.

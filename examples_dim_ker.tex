\documentclass[a4paper,14pt]{article} %размер бумаги устанавливаем А4, шрифт 12пунктов
\usepackage[T2A]{fontenc}
\usepackage[utf8]{inputenc}
\usepackage[english,russian]{babel} %используем русский и английский языки с переносами
\usepackage{amssymb,amsfonts,amsmath,mathtext,cite,enumerate,float,amsthm} %подключаем нужные пакеты расширений
\usepackage[unicode,colorlinks=true,citecolor=black,linkcolor=black]{hyperref}
%\usepackage[pdftex,unicode,colorlinks=true,linkcolor=blue]{hyperref}
\usepackage{indentfirst} % включить отступ у первого абзаца
\usepackage[dvips]{graphicx} %хотим вставлять рисунки?
\graphicspath{{illustr/}}%путь к рисункам

\makeatletter
\renewcommand{\@biblabel}[1]{#1.} % Заменяем библиографию с квадратных скобок на точку:
\makeatother %Смысл этих трёх строчек мне непонятен, но поверим "Запискам дебианщика"

\usepackage{geometry} % Меняем поля страницы.
\geometry{left=2cm}% левое поле
\geometry{right=1cm}% правое поле
\geometry{top=2cm}% верхнее поле
\geometry{bottom=2cm}% нижнее поле

\renewcommand{\theenumi}{\arabic{enumi}}% Меняем везде перечисления на цифра.цифра
\renewcommand{\labelenumi}{\arabic{enumi}}% Меняем везде перечисления на цифра.цифра
\renewcommand{\theenumii}{.\arabic{enumii}}% Меняем везде перечисления на цифра.цифра
\renewcommand{\labelenumii}{\arabic{enumi}.\arabic{enumii}.}% Меняем везде перечисления на цифра.цифра
\renewcommand{\theenumiii}{.\arabic{enumiii}}% Меняем везде перечисления на цифра.цифра
\renewcommand{\labelenumiii}{\arabic{enumi}.\arabic{enumii}.\arabic{enumiii}.}% Меняем везде перечисления на цифра.цифра

\sloppy


\renewcommand\normalsize{\fontsize{14}{25.2pt}\selectfont}

\begin{document}
% !!!
% Здесь начинается реальный ТеХ-код
% Всё, что выше - беллетристика

\paragraph{Тезис.}
Существует оператор с конечномерным ядром, для которого не существует инвариантных банаховых пределов.
Существует оператор с бесконечномерным ядром, для которого существует инвариантный банахов предел.

\paragraph{Пример.}

Пусть для $x = (x_1, x_2, ..., x_n, ...)\in l_\infty$
\begin{equation}
	Ax = (x_1, 0, x_2, 0, x_3, 0, x_4, 0, ...).
\end{equation}
Очевидно, что $\ker A = {0}$.
Покажем, что банаховых пределов, инвариантых относительно $A$, не существует.
Предположим противное.
Пусть $B\in\mathfrak{B}$, $BA = B$.
Пусть $\mathbb{I} = (1, 1, 1, 1, 1, 1, ...)$.
Очевидно, что
%
\begin{equation}
	\frac{n-1}{2}\leqslant \sum_{k=m+1}^{m+n} (A\mathbb{I})_k \leqslant \frac{n+1}{2},
\end{equation}
откуда
\begin{equation}
	\frac{n-1}{2n}\leqslant \frac{1}{n}\sum_{k=m+1}^{m+n} (A\mathbb{I})_k \leqslant \frac{n+1}{2n}.
\end{equation}
По теореме Лоренца
\begin{equation}
	BA\mathbb{I} = 
	\lim_{n\to\infty} \sum_{k=m+1}^{m+n} (A\mathbb{I})_k = \frac{1}{2}
\end{equation}

Однако по определению банахова предела
\begin{equation}
	B\mathbb{I} = 1 \neq \frac{1}{2} = BA\mathbb{I}
\end{equation}
Пришли к противоречию, следовательно, банаховых пределов, инвариантных относительно $A$, не существует.


\end{document}
